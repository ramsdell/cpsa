\documentclass[12pt]{report}
\usepackage{amssymb}
\usepackage{amsmath}
\usepackage{amsthm}
\usepackage{url}
\usepackage[matrix,arrow,curve]{xy}
\usepackage{makeidx}
% Include this for a hyperlinked version of the document
% \usepackage{hyperref}
\theoremstyle{definition}
\newtheorem{defn}{Definition}[chapter]
\newtheorem{thm}{Theorem}[chapter]
\newtheorem{lem}[thm]{Lemma}
\newtheorem{cor}[thm]{Corollary}
\newtheorem{conj}[thm]{Conjecture}
\newtheorem{prop}[thm]{Proposition}
\newcommand{\cpsa}{\textsc{cpsa}}
\newcommand{\version}{4.2.4}
\newcommand{\cpsacopying}{\begingroup
  \renewcommand{\thefootnote}{}\footnotetext{{\copyright} 2010 The
    MITRE Corporation.  Permission to copy without fee all or part of
    this material is granted provided that the copies are not made or
    distributed for direct commercial advantage, this copyright notice
    and the title of the publication and its date appear, and notice
    in given that copying is by permission of The MITRE
    Corporation.}\endgroup}
\newcommand{\pov}{\textsc{pov}}
\newcommand{\cow}{\textsc{cow}}
\newcommand{\cn}[1]{\ensuremath{\operatorname{\mathsf{#1}}}}
\newcommand{\dom}[1]{\ensuremath{\operatorname{\mathbf{#1}}}}
\newcommand{\fn}[1]{\ensuremath{\operatorname{\mathit{#1}}}}
\newcommand{\seq}[1]{\ensuremath{\langle#1\rangle}}
\newcommand{\enc}[2]{\ensuremath{\{\!|#1|\!\}_{#2}}}
\newcommand{\hash}[1]{\ensuremath{\##1}}
\newcommand{\comp}[2]{{\mathord{#1}}\circ{\mathord{#2}}}
\newcommand{\sembrack}[1]{[\![#1]\!]}
\newcommand{\semfn}[1]{\mathcal{#1}}
\newcommand{\sem}[2]{\semfn{#1}\sembrack{#2}}
\newcommand{\mcsu}{\ensuremath{\mathcal{C}}}
\newcommand{\probs}{\ensuremath{\mathcal{E}}}
\newcommand{\srt}[1]{\ensuremath{\mathsf{#1}}}
\newcommand{\eqq}{\stackrel{?}{=}}
\newcommand{\inbnd}{\mathord -}
\newcommand{\outbnd}{\mathord +}
\newcommand{\nat}{\mathbb{N}}
\newcommand{\all}[1]{\mathop{\forall#1}}
\newcommand{\some}[1]{\mathop{\exists#1}}
\newcommand{\baseterms}{\mathbb{B}}
\newcommand{\cons}{\mathbin{::}}
\newcommand{\ith}{\imath^\mathrm{th}}
\newcommand{\jth}{\jmath^\mathrm{th}}
\newcommand{\append}{\mathbin{{}^\smallfrown}}
\newcommand{\prefix}[2]{#1\dagger#2}
\newcommand{\orig}{\mathcal{O}}
\newcommand{\gain}{\mathcal{G}}
%\newcommand{\pow}[1]{\mathcal{P}(#1)}
%\newcommand{\pow}[1]{\wp(#1)}
\newcommand{\pow}[1]{2^{#1}}
\newcommand{\termat}{\mathbin{@}}
\newcommand{\idsigma}{\sigma_{\mathrm{id}}}
\newcommand{\idphi}{\phi_{\mathrm{id}}}
\newcommand{\kprec}[1]{\mathbin{\prec_{#1}}}
\newcommand{\terms}{\ensuremath{\mathcal{T}}}
\newcommand{\alg}[1]{\ensuremath{\mathfrak#1}}
\newcommand{\ops}[1]{\ensuremath{\mathbb#1}}
\newcommand{\bundle}{\ensuremath{\mathcal{B}}}
\newcommand{\der}[3]{\ensuremath{\dctx{#2}{#3}\vdash#1}}
\newcommand{\dctx}[2]{\ensuremath{#1:#2}}
\newcommand{\flow}[3]{\ensuremath{#1,#2\rhd#3}}
\newcommand{\infer}[2]{\begin{array}{c}#1\\\hline#2\end{array}}
\hyphenation{pre-skel-e-ton}
\newcommand{\homomorphism}[1]{\stackrel{#1}{\longmapsto}}
\newcommand{\reduction}[1]{\stackrel{#1}{\longrightarrow}}
\newcommand{\longtwoheadrightarrow}{\relax
\smash{\relbar\joinrel\twoheadrightarrow}\vphantom\rightarrow}
\newcommand{\setreduction}[1]{\stackrel{#1}{\longtwoheadrightarrow}}
\newcommand{\uniqat}{\mbox{\sf uniq-at}}
\newcommand{\strprec}{\mbox{\sf str-prec}}


\hyphenation{pro-to-skel-e-ton}
\hyphenation{pro-to-role}

% This environment is used for comments about this draft.
\newenvironment{note}{\itshape\par}{}

\newcommand{\inv}[1]{\ensuremath{\cn{inv}(#1)}}
\newcommand{\gen}{\ensuremath{\cn{g}}}
\newcommand{\expr}[2]{\ensuremath{{#1}^{#2}}}
\newcommand{\bca}{\textsc{bca}}
\newcommand{\alga}{\alg{A}}
\newcommand{\algbara}{\bar{\alg{A}}}
\newcommand{\tr}{\ensuremath{\mathfrak C}}
\newcommand{\base}{\ensuremath{\mathfrak B}}
\newcommand{\ints}{\ensuremath{\mathbb Z}}
\newcommand{\solve}[1]{\stackrel{#1}{\longtwoheadrightarrow}}
\newcommand{\sntchmsg}{\fn{sntchmsg}}
\newcommand{\pubmsg}{\fn{pubmsg}}
\newcommand{\auth}{\fn{auth}}
\newcommand{\conf}{\fn{conf}}
\newcommand{\avoid}{\fn{avoid}}
\newcommand{\dcmp}{\fn{dcmp}}
\newcommand{\carpos}{\fn{carpos}}
\newcommand{\esc}{\fn{esc}}
\newcommand{\aug}{\fn{aug}}
\newcommand{\crit}{\fn{crit}}
\newcommand{\critpos}{\fn{critp}}
\newcommand{\anc}{\fn{anc}}
\newcommand{\mcow}{\fn{cow}}
\newcommand{\mncow}{\fn{ncow}}
\newcommand{\sdom}{\fn{Dom}}
\newcommand{\sran}{\fn{Ran}}
\newcommand{\svars}{\fn{Vars}}
\newcommand{\evt}{\fn{evt}}
\newcommand{\msg}{\fn{msg}}
\newcommand{\role}{\mathsf{r}}
\newcommand{\skel}{\mathsf{k}}
\newcommand{\rl}{\fn{rl}}
\newcommand{\inst}{\mathsf{i}}
\newcommand{\nodes}{\fn{nodes}}
\newcommand{\lsn}{\fn{lsn}}
\newcommand{\init}{\fn{init}}
\newcommand{\resp}{\fn{resp}}
\newcommand{\form}{\mathcal{F}}
\newcommand{\sent}{\mathcal{S}}

\makeindex

\title{The CPSA Specification:\\
A Reduction System for Searching for Shapes in Cryptographic Protocols}
\author{John D.~Ramsdell\qquad Joshua D.~Guttman\\
  \quad Moses D.~Liskov\qquad Paul D.~Rowe\\
  The MITRE Corporation\\ CPSA Version \version}

\begin{document}
\maketitle
\cpsacopying

\tableofcontents

\begin{abstract}
We describe a term reduction system that enumerates all essentially
different executions possible for a cryptographic protocol.  We call
them the \emph{shapes} of the protocol.  Naturally occurring protocols
have only finitely many, indeed very few shapes.  Authentication and
secrecy properties are easy to determine from them, as are attacks and
anomalies.  Our Cryptographic Protocols Shapes Analyzer ({\cpsa})
program is a direct implementation of the reduction system described
within, and the form of the reduction system is partially determined
by the implementation.
\end{abstract}

\chapter{Introduction}

The Cryptographic Protocol Shapes Analyzer ({\cpsa}) attempts to
enumerate all essentially different executions possible for a
cryptographic protocol.  We call them the \emph{shapes} of the
protocol.  Naturally occurring protocols have only finitely many,
indeed very few shapes.  Authentication and secrecy properties are
easy to determine from them, as are attacks and anomalies.

The shapes analysis is performed within a pure Dolev-Yao model.  The
{\cpsa} program reads a sequence of problem descriptions, and prints
the steps it used to solve each problem.  For each input problem,
{\cpsa} is given some initial behavior, and it descovers what shapes
are compatible with it.  Normally, the initial behavior is from the
point of view of one participant.  The analysis reveals what the other
participants must have done, given the participant's view.

This document specifies the {\cpsa} program using a term reduction
system.  Chapter~\ref{chp:order-sorted mesg algs} describes message
algebras as order-sorted quotient term algebras.
Chapter~\ref{chp:strand spaces and bundles} presents an
implementation-oriented view of strand spaces.
Chapter~\ref{chp:adversary model} details the model of the adversary.
The formal definition of a partial run of a protocol is called a
skeleton, and is introduced in Chapter~\ref{chp:skeletons}.

The term reduction systems used to specify the algorithm is presented
in Chapter~\ref{chp:algorithms}, and the primitive reduction rules are
in Chapter~\ref{chp:prim preskel ops}.  The rules used to transform
terms called preskeletons into skeletons are in
Chapter~\ref{chp:preskeleton reduction}.

The algorithm used to model adversarial behavior is in
Chapter~\ref{chp:penetrator derivable}.  The algorithms used to infer
what else must have happened given a partial description of a run of
a protocol as skeleton is in Chapters~\ref{chp:cow} and~\ref{chp:auth
  tests}.  Chapters~\ref{chp:generalization} and~\ref{chp:collapsing}
find most general descriptions of {\cpsa} answers---the shapes.
Finally, Chapter~\ref{chp:skeleton reduction} assembles reduction
rules into one system that specifies the {\cpsa} program.

Appendix~\ref{chp:penetrator non-origination} describes an extension
to Strand Space theory that models passwords and related concepts.
Appendix~\ref{chp:traces} details the sense in which a protocol role
can be viewed as an abstraction of a program.
Appendix~\ref{chp:logic} describes a formula in the language of
order-sorted first-order logic for each problem and its shapes found
by {\cpsa}.  The formula is called a shape analysis sentence.  The
formula is modeled by all skeletons that describe full runs of a
protocol when {\cpsa} finds all the shapes for the problem.

{\cpsa}'s search is based on a high-level algorithm that was claimed
to be complete, i.e.\@ every shape can in fact be found in a finite
number of steps~\cite{DoghmiGuttmanThayer07,Guttman11}.  Further
theoretical work~\cite{cpsatheory11} showed classes of executions that
are not found by the algorithm, however it also showed that every
omitted execution requires an unnatural interpretation of a protocol's
roles.  Hence the algorithm is complete relative to natural role
semantics.  See~\cite[Appendix~B]{cpsaprimer09} for more on omitted
executions.

A {\cpsa} release includes two other documents, the {\cpsa}
Design~\cite{cpsadesign09} and the {\cpsa} Primer~\cite{cpsaprimer09}.
The design document describes details of the {\cpsa}
implementation that would clutter this one.  It should be read by
anyone interesting in reading and modifying the source code.  The
{\cpsa} Primer provides an overview of {\cpsa}, and is worth
reading before this document is approached.

\section{Notation}\label{sec:notation}

A finite sequence\index{sequence} is a function from an initial
segment of the natural numbers.  The length of a sequence~$X$ is~$|X|$,
and sequence~$X=\seq{X(0),\ldots,X(n-1)}$ for $n=|X|$.  Alternatively,
$\seq{x_0,x_1,\ldots,x_{n-1}} =x_0\cons x_1\cons\ldots\cons
x_{n-1}\cons\seq{}$.  If~$S$ is a set, then~$S^\ast$ is the set of finite
sequences of~$S$, and~$S^+$ is the non-empty finite sequences of~$S$.
The concatenation of sequences~$X_0$ and~$X_1$ is~$X_0\append X_1$.
The prefix of sequence~$X$ of length~$n$ is~$\prefix{X}{n}$.

Generally, when discussing terms, a lowercase Latin letter is used to
denote a term, and an uppercase Latin letter is used to denote a set
of terms or a sequence of terms.

\chapter{Order-Sorted Message Algebras}\label{chp:order-sorted mesg algs}

{\cpsa} models a message by an equivalence class of terms over a
signature. A sort system is used to classify messages.  {\cpsa}
depends on the sort system to allow it to treat a variable that
represents an asymmetric key differently from a variable that
represents an arbitrary message.  In particular, {\cpsa} uses
order-sorted quotient term algebras~\cite{GoguenMeseguer92} for
message algebras.  This formalism enables the use of well-known
algorithms for unification and matching in the presences of equations
and sorts~\cite{BaaderSnyder01}.

This paper makes no attempt to provide a general introduction to
order-sorted quotient term algebras.  We use a message algebra called
the Basic Crypto Algebra (\bca), which is the main algebra used by
{\cpsa}.

\begin{figure}
\begin{center}
Sorts: \dom{name}, \dom{text}, \dom{data}, \dom{skey}, \dom{akey} $<$ \dom{mesg}\\[1ex]
Base sorts: \dom{name}, \dom{text}, \dom{data}, \dom{skey}, \dom{akey}\\
Carried positions: $\bullet$ denotes a carried position.\\[1ex]
\begin{tabular}{@{}ll}
$\enc{\bullet}{(\cdot)}\colon\dom{mesg}\times\dom{mesg}\rightarrow\dom{mesg}
$&Encryption\\
$\hash{(\cdot)}\colon\dom{mesg}\rightarrow\dom{mesg}
$&Hashing\\
$(\bullet,\bullet)\colon\dom{mesg}\times\dom{mesg}\rightarrow\dom{mesg}$
&Pairing\\
``\ldots''$\colon{}\dom{mesg}$& Tag constants\\
$K_{(\cdot)}\colon\dom{name}\rightarrow\dom{akey}$
&Public key of name\\
$(\cdot)^{-1}\colon\dom{akey}\rightarrow\dom{akey}$
&Inverse of key\\
$\cn{ltk}(\cdot,\cdot)\colon\dom{name}\times\dom{name}\rightarrow\dom{skey}$
& Long term key
\end{tabular}\\[1ex]
Equation: $(x^{-1})^{-1} \approx x$ for $x\colon\dom{akey}$
\end{center}
\caption{Basic Crypto Signature and Equation}\label{fig:bcs}
\end{figure}

There are six {\bca} sorts: $\dom{mesg}$, \index{sorts}the sort of
all messages, $\dom{skey}$, the sort of symmetric keys,
$\dom{akey}$, the sort of asymmetric keys, $\dom{name}$, the sort of
participant names, and $\dom{text}$ and $\dom{data}$ for ordinary
values.  Sort $\dom{mesg}$ is sometimes written as~$\top$ and the
other sorts are called \index{base sorts}\emph{base sorts}.
All base sorts are subsorts of $\dom{mesg}$.  The function
symbols, or \index{operations}\emph{operations}, used to form terms
are given by the signature in Figure~\ref{fig:bcs}.

Each variable~$x$ used to form a term has a unique sort~$s$, written
$x\colon s$.  Variable set~$X$\index{variable set} is an indexed set
of sets of variables, $X_s=\{x\mid x\colon s\}$.  For {\bca},
$X_{\dom{mesg}}$, $X_{\dom{skey}}$, $X_{\dom{akey}}$,
$X_{\dom{name}}$, $X_{\dom{text}}$, and $X_{\dom{data}}$ partition
the set of variables in~$X$.  By abuse of notation, at times, we
write~$X$ for the set of variables in~$X$.

\begin{figure}
$$\begin{array}{r@{{}={}}l}
\alga_{\dom{skey}}&\{\{x\}\mid x \in X_{\dom{skey}}\} \cup \{\{\cn{ltk}(a,b)\}\mid a \in X_{\dom{name}}, b \in X_{\dom{name}}\}\\
\alga_{\dom{akey}}&\begin{array}[t]{@{}l}
\{\{x^{-2n}\mid n\in\nat\}\mid x\in X_{\dom{akey}}\}\\
\quad{}\cup\{\{x^{-2n-1}\mid n\in\nat\}\mid x\in X_{\dom{akey}}\}\\
\quad{}\cup\{\{K_x^{-2n}\mid n\in\nat\}\mid x\in X_{\dom{name}}\}\\
\quad{}\cup\{\{K_x^{-2n-1}\mid n\in\nat\}\mid x\in X_{\dom{name}}\}\\
\end{array}\\
\alga_{\dom{name}}&\{\{x\}\mid x \in X_{\dom{name}}\}\\
\alga_{\dom{text}}&\{\{x\}\mid x \in X_{\dom{text}}\}\\
\alga_{\dom{data}}&\{\{x\}\mid x \in X_{\dom{data}}\}\\
\base&\alga_{\dom{skey}} \cup \alga_{\dom{akey}} \cup \alga_{\dom{name}} \cup \alga_{\dom{text}} \cup \alga_{\dom{data}}\\
\alga^0&\base\cup \{\{x\}\mid x\in X_{\dom{mesg}}\} \cup
\{\{x\}\mid x \hbox{ is a tag constant }\}\\
\alga^{n+1}&\alga^n
\begin{array}[t]{@{}l}
{}\cup\{\{(t_0,t_1)\mid t_0\in T_0, t_1\in T_1\}\mid
 T_0\in\alga^n,T_1\in\alga^n\}\\
{}\cup\{\{\enc{t_0}{t_1}\mid t_0\in T_0, t_1\in T_1\}\mid
 T_0\in\alga^n,T_1\in\alga^n\}\\
{}\cup\{\{\hash{t}\mid t\in T\}\mid T\in\alga^n\}
\end{array}\\
\alga = \alga_{\dom{mesg}}&\bigcup_{n\in\nat}\alga^n
\end{array}$$
\caption{{\bca} Messages~$\alga$ and Atoms~$\base$}\label{fig:algebra}
\end{figure}

The Basic Crypto Quotient Term Algebra~$\alga$ generated by variable
set~$X$ is displayed in Figure~\ref{fig:algebra}.  The union of the
messages in~$\alga$ is set of terms generated by~$X$, and~$\alga$
partitions the set of terms into a set of equivalence classes induced
by the equations.  Terms~$t_0$ and~$t_1$ are equivalent, written
$t_0\equiv t_1$, iff $t_0\in T\land t_1\in T$ for some $T\in\alga$.
The canonical representative of a message is the $t$ in $\{t'\mid
t'\equiv t\}$ with the fewest occurrences of the $(\cdot)^{-1}$
operation.

Keys, names, data, and texts in the algebra are called
\index{atoms}\emph{atoms} and are members of~$\base$.  We write
$t\colon B$ iff $\{t'\mid t'\equiv t\}\in\base$.  Note that encryption
is defined with an encryption key of sort $\dom{mesg}$.  When the
encryption key is of sort $\dom{akey}$ this is meant to model
asymmetric encryption: otherwise, this models symmetric encryption.
Note that even complex messages such as encryptions can be used as
encryption keys in the symmetric sense.

To find the decryption key associated with an encryption, one must
exclude the case in which the key is a variable of sort \dom{mesg},
as there is no way to determine if the encryption operation denotes
symmetric or asymmetric encryption.  Therefore, the decryption
key\index{inverse key} associated with encryption key~$t$
is~$\fn{inv}(t)$.

$$\fn{inv}(t) =\left\{
\begin{array}{ll}
\cn{invk}(t)&\mbox{if $t\colon\dom{akey}$;}\\
\mbox{undefined}&\mbox{if $t$ is a variable of sort $\dom{mesg}$;}\\
t&\mbox{otherwise.}
\end{array}\right.$$

An important property possessed by the algebra is that for all
$T\in\alga$, if there are any encryptions in $T$ then all members of
$T$ are encryptions.  As a result, a message can be identified as
representing an encryption and if it is, decomposed into its plaintext
and its decryption key.  This property is a consequence of the fact
that equations relate atoms, not arbitrary messages.  A similar
property holds for pairs and hashes.  A hash\index{hash} is treated as
a kind of encryption in which the term that is hashed is the
encryption key.

We write $\alga_X$ when it is important to identify the variable
set~$X$ that generates the algebra.  Given two variable sets~$X$
and~$Y$, a \index{substitution}\emph{substitution} is an order-sorted
map~$\sigma\colon X\rightarrow\alga_Y$ such that $\sigma(x)\neq x$ for
only finitely many elements of~$X$.  For a substitution~$\sigma$, the
\index{domain}\index{domain}\emph{domain} is the set of variables
$\sdom(\sigma)=\{x\mid \sigma(x)\neq x\}$ and the
\index{range}\emph{range} is the set $\sran(\sigma)=\{\sigma(x)\mid
x\in\sdom(\sigma)\}$.  Substitution~$\sigma_0$ is \index{more general
  substitution}\emph{more general than}~$\sigma_1$,
written~$\sigma_0\unlhd\sigma_1$, if there exists a
substitution~$\sigma_2$ such that
$\all{x}\sigma_1(x)\equiv\sigma_2(\sigma_0(x))$.  Given a
substitution~$\sigma\colon X\rightarrow\alga_Y$, the unique
homomorphism\index{homomorphism!algebra} $\sigma^\ast\colon
\alga_X\rightarrow\alga_Y$ induced by~$\sigma$ is also
denoted~$\sigma$.

A \index{position}\emph{position}~$p$ is a finite sequence of natural
numbers.  The term in~$t$ that \emph{occurs at}~$p$, written~$t\termat
p$, is:
$$\begin{array}{l}
t\termat\seq{}=t;\\
(t_0,t_1)\termat i\cons p=t_i\termat p\mbox{ for $i\in\{0,1\}$};\\
\enc{t_0}{t_1}\termat i\cons p=t_i\termat p\mbox{ for
  $i\in\{0,1\}$};\\
t^{-1}\termat 0\cons p=t\termat p.
\end{array}$$
A term~$t$ \index{occurs in}\emph{occurs in} term~$t'$ if $t=t'\termat
p$ for some~$p$.  A message~$T$ \emph{occurs in} message~$T'$ if the
canonical representative of~$T$ occurs in the canonical representative
of~$T'$.

A carried term is one that can be extracted from a message reception
assuming plaintext is extractable from encryptions.  The positions at
which term $t$ is carried in $t'$ is $\carpos(t,t')$\index{carried
  positions}\index{carpos@\carpos}\label{def:carried positions},
where
$$\carpos(t,t')=\left\{
\begin{array}{ll}
\{\seq{}\}&\mbox{if $t'\equiv t$, else}\\
\multicolumn{2}{l}{\{0\cons p \mid
p\in\carpos(t,t_1)\}}\\
& \mbox{if $t'=\enc{t_0}{t_1}$, else}\\
\multicolumn{2}{l}{\{i\cons p \mid
i\in\{0,1\}, p\in\carpos(t,t_i)\}} \\
& \mbox{if $t'=(t_0,t_1)$ else}\\
\emptyset&\mbox{otherwise.}
\end{array}\right.$$

Term~$t$ \index{carries}\emph{carries}~$t'$ if $\carpos(t',t)$ is not
empty, and $t'\sqsubseteq t$ when~$t'$ is carried by~$t$.  Note that
for all terms $t_0,t_1,t'_0,t'_1$, if $t_0\equiv t_1$ and $t'_0\equiv
t'_1$, then $\carpos(t_0,t'_0)= \carpos(t_1,t'_1)$.  We write $t'
\sqsubseteq_p t$ when $p \in \carpos(t',t)$ and $t \termat p \equiv
t'$.

In what follows, we will often conflate a term with the message of
which it is a member, and use lowercase letters to denote both.

\section{Channel Messages}\index{channel message}

As of {\cpsa} version 4.2, messages can optionally be transported on a
channel.  For $t\in\alga_X$, a plain message is written $[t]$, and
when the message is bound to channel~$c$, it is written $[c,t]$.

The channel message algebra adds two sorts to the underlying message
algebra, \dom{chan} and \dom{chmsg}, and two operations to the
signature:
\[\begin{array}{r@{{}:{}}ll}
   [\cdot]&\dom{mesg}\rightarrow\dom{chmsg}&\mbox{Plain message}\\ \relax
   [\cdot,\cdot]&\dom{chan}\times\dom{mesg}\rightarrow\dom{chmsg}
   &\mbox{Channel message}
\end{array}\]
Let $X$ be a {\bca} variable set, and $\bar X$ be a variable set that
adds channel variables to $X$.  If $\alga_X$ is the algebra associated
with sort \dom{mesg}, then $\algbara_{\bar X}$ is the algebra
associated with sort \dom{chmsg}, the Basic Crypto Channel Algebra
generated from $\bar X$.

Positions are extended to channel messages as follows:
\[\begin{array}{l}
m\termat\seq{}=m;\\ \relax
[t]\termat 0\cons p=t\termat p;\\ \relax
[c,t]\termat 1\cons p=t\termat p.
\end{array}\]
The set of carried positions is extended to channel messages in the
obvious way.  The plain message $[t]$ will often be written as $t$
when the context disambiguates the reference.

\chapter{Strand Spaces and Bundles}\label{chp:strand spaces and bundles}

When using strand space theory, one normally hypothesizes the
existence of a single global strand space.  This is a very reasonable
assumption for theoretical analysis, but from the point view of an
implementer, it turns out that it is better to assume there are many
local strand spaces and the design specification task is to describe
the relations between these local spaces.  Our reformulation of strand
space notation provides an implementation oriented way of describing
the concept of a local strand space, and a direct link between from
algorithm specification to the data structures used in the
implementation.

A run of a protocol is viewed as an exchange of messages by a finite
set of local sessions of the protocol.  Each local session is called a
\index{strand}\emph{strand}.  The behavior of a strand, its
\index{trace}\emph{trace}, is a sequence of messaging events.  An
\index{event}\emph{event} is either a message transmission or a
reception.  Outbound channel message $m\in\algbara_X$ is written
as~$\outbnd m$, and inbound channel message~$m$ is written as~$\inbnd
m$.  The set of traces over $\algbara_X$ is $\tr_X=(\pm\algbara_X)^+$.  A
message \index{originates}\emph{originates} in a trace if it is
carried by some event and the first event in which it is carried is
outbound.  A message is \index{gained}\emph{gained} by a trace if it
is carried by some event and the first event in which it is carried is
inbound.  A message is \index{acquired}\emph{acquired} by a trace if
it first occurs in a reception event and is also carried by that
event.

Abstractly, a strand space is a multiset of traces, but since we wish
to name each element, a \index{strand space}\emph{strand
  space}~$\Theta_X$ over algebra~$\algbara_X$ is defined to be a sequence
of traces in $\tr_X$.  A strand~$s$ is a member of the domain of
$\Theta_X$, and its trace is $\Theta_X(s)$.

Message events occur at nodes in a strand space.  For each strand~$s$,
there is a node for every event in~$\Theta(s)$.  The
\index{nodes}\emph{nodes} of strand space $\Theta$ are $\{(s,i)\mid
s\in\sdom(\Theta), 0\leq i< |\Theta(s)|\}$, the event at a node is
\index{evt@\evt}$\evt_\Theta(s,i)=\Theta(s)(i)$, and the message at a
node is \index{msg@\msg}$\msg_\Theta(s,i) = m$ such that
$\evt_\Theta(s,i) = \pm m$.  Just as a position names a subterm within
another term, a strand names a trace within a strand space, and a node
names an event in a strand space.  The relation~$\Rightarrow$ defined
by $\{(s,i-1)\Rightarrow(s,i)\mid s\in\sdom(\Theta), 1\leq
i<|\Theta(s)|\}$ is called the \index{strand succession}\emph{strand
  succession relation}.

A \index{bundle}\emph{bundle} in strand space $\Theta$ is a finite
directed acyclic graph $\Upsilon(\Theta,\to)$, where the vertices are
the nodes of $\Theta$, and an edge represents
communication~($\rightarrow$) or strand succession~($\Rightarrow$).
For \index{communication}communication, if $n_0\rightarrow n_1$, then
there is a channel message~$m$ such that~$\evt_\Theta(n_0)=\outbnd m$
and~$\evt_\Theta(n_1)=\inbnd m$.  For each reception node~$n_1$, there
is a unique transmission node~$n_0$ with $n_0\rightarrow n_1$.

Each acyclic graph has a transitive irreflexive relation~$\prec$ on
its vertices.  The relation specifies the causal ordering of nodes in
a bundle.  An transitive irreflexive binary relation is also called a
\index{strict partial order}\index{partial order!strict}strict partial
order.

An atom \index{uniquely originates}\emph{uniquely originates} in a
bundle if it originates in the trace of exactly one strand.  An atom
is \index{non-originating}\emph{non-originating} in a bundle if it
originates on no strand, but each of its variables occurs in some
strand's trace.  A channel is \index{authenticated
  channel}\emph{authenticated} if only regular strands in a bundle use
it to transmit messages.  A channel is \index{confidential
  channel}\emph{confidential} if only regular strands in a bundle use
it to receive messages.

In a run of a protocol, the behavior of each strand is constrained by
a role in a protocol.  Adversarial strands are constrained by roles as
are non-adversarial strands.  What follows is the definition of a role
before channels and channel messages were introduced into {\cpsa},
followed by a discussion of the additions caused by channels.

A protorole over $\algbara_Y$ is
$\role_Y(C,N,U)$, where $C\in\tr_Y$, $N\subseteq\base_Y$, and
$U\subseteq\base_Y$.  The trace of the role is~$C$, its
non-origination assumptions are~$N$, and its unique origination
assumptions are~$U$.  A protorole is a \index{role}\emph{role} if (1)
$t\in N$ implies $t$ is not carried in $C$, and all variables in $N$
occur in $C$, (2) $t\in U$ implies $t$ originates in $C$, (3) if
variable~$x$ occurs in $C$ then $x$ is an atom or it is acquired in
$C$, and (4) the trace of a role may not match the pattern
$\seq{\inbnd m, \outbnd m,\ldots}$.  This is to ensure
that listeners, which are introduced on Page~\pageref{def:listeners},
cannot be confused with protocol constrained strands.
A \index{protocol}\emph{protocol} is a set of roles.  Let
\index{Vars@\svars}$\svars(P)$ be the set of variables that occur in
the traces of the roles in protocol~$P$.

A bundle~$\Upsilon(\Theta_X,\to)$ is a \index{run of
  protocol}\emph{run of protocol} $P$ if there is a role mapping
$\rl\colon\Theta_X\to P$ that satisfies properties for each
$s\in\sdom(\Theta_X)$.  Assuming $\rl(s)=\role_Y(C,N,U)$ and $X$ and
$Y$ share no variables, and let $h=|\Theta_X(s)|$, the properties are
(1) $h\leq|C|$, (2) there is a
homomorphism~$\sigma\colon\algbara_Y\to\algbara_X$ such that
$\comp{\sigma}{\prefix{C}{h}}=\Theta_X(s)$, (3) $\sdom(\sigma)$ is the
set of variables that occur in~$\prefix{C}{h}$, (4) if the variables
in $t\in N$ occur in $\sdom(\sigma)$, then $\sigma(t)$ is
non-originating in $\Upsilon(\Theta_X,\to)$, and (5) if $t\in U$
originates at index~$i$ in~$C$, and $i<h$, then $\sigma(t)$
uniquely originates in $\Upsilon(\Theta_X,\to)$ at node $(s,i)$.
Origination assumptions in bundles specified by roles are called
\index{inherited origination assumptions}\emph{inherited origination
  assumptions}.

The introduction of channels adds two new fields to a role, a set of
channels assumed to be authenticated, and a set of channels assumed to
be confidential.  For a bundle to be a run of a protocol, the
inherited channel assumptions must be honored by the bundle.

\chapter{Adversary Model}\label{chp:adversary model}

A fixed set of penetrator roles encodes the adversary model associated
with a message algebra.  For the Basic Crypto Algebra, there are nine
roles.  Each role makes no origination assumptions, and the trace of
each role is given in Figure~\ref{fig:pen}.  The first line of the
figure specifies many traces, one for each base sort, and a trace for
each tag.

\begin{figure}
$$\begin{array}{lll}
\mbox{Create}(z\colon B)&
\seq{\outbnd z}&\seq{\outbnd\mbox{``\ldots''}}\\
\mbox{Pair}(x,y\colon\top)&
\seq{\inbnd x,\inbnd y,\outbnd(x, y)}&
\seq{\inbnd(x, y),\outbnd x,\outbnd y}\\
\mbox{Encrypt}(x,y\colon\top)&
\seq{\inbnd x,\inbnd y,\outbnd\enc{x}{y}}&
\seq{\inbnd\enc{x}{y},\inbnd \fn{inv}(y),\outbnd x}\\
\mbox{Hash}(x\colon\top)&
\seq{\inbnd x,\outbnd\hash{x}}\\
\mbox{Chan}(c\colon\mathit{Ch},x\colon\top)
&\seq{\inbnd x, \outbnd[c,x]}&\seq{\inbnd[c,x],\outbnd x}
\end{array}$$
\caption{Basic Crypto Algebra Penetrator Role Traces}\label{fig:pen}
\end{figure}

A strand exhibits non-adversarial behavior when its role is not a
penetrator role.  A non-adversarial strand is called a \index{regular
  strand}\emph{regular} strand as is its role.

The penetrator cannot use a non-originating atom to encrypt or decrypt
a message, because every key it uses must be carried in a message.
Consider a uniquely originating atom that originates on a regular
strand.  The penetrator cannot make the atom using a create role,
because the atom would originate in more than one trace.  Therefore,
the penetrator can use a uniquely originating atom to encrypt or
decrypt a message only if it is transmitted by a regular strand
unprotected by encryption.

When a channel is authenticated, the penetrator cannot send a message
on it.  When a channel is confidential, the penetrator cannot receive
a message on it.

\chapter{Skeletons}\label{chp:skeletons}

The details of penetrator behavior are abstracted away when performing
protocol analysis.  The abstracted description of a bundle is called a
realized skeleton, which is defined using a protoskeleton.  A
\emph{protoskeleton} over $\algbara_X$ is
$\skel_X(\rl,P,\Theta_X,\prec,N,U)$, where $\rl\colon\sdom(\Theta_X)\to P$ is
a role map, the sets~$X$ and $\svars(P)$ are disjoint,
$\Theta_X$ is a sequence of traces in $\tr_X$, $\prec$ is a relation
on the nodes in $\Theta_X$, $N\subseteq\base_X$ are its
non-origination assumptions, and $U\subseteq\base_X$ are its unique
origination assumptions.  Unlike a strand space, the sort of a
variable in~$X$ need not be a base sort.

The introduction of channels adds two new fields to a skeleton, a set
of channels assumed to be authenticated, and a set of channels assumed
to be confidential.  The implications of the additions is
straightforward, and we simply retain the original description of
skeletons.

Assume the strands in bundle $\Upsilon(\Theta_X,\to)$ have been
permuted so that regular strands precede penetrator strands in
sequence~$\Theta_X$, and $\rl$ demonstrates the bundle is a run of
protocol~$P$.  Let~$P'$ be~$P$ without penetrator roles.  Skeleton
$\skel_X(\rl',P',\Theta'_X,\prec,N,U)$ \index{realized
  skeleton}\index{skeleton!realized}\emph{realizes} the bundle if
$\rl'$ and $\Theta'_X$ are the truncations of $\rl$ and $\Theta_X$
respectively that omit penetrator strands from their domains, $\prec$
is the transitive irreflexive relation associated with the bundle
without penetrator nodes, $N$ is the set of non-originating atoms with
variables that occur in $\Theta'_X$, and $U$ is the set of atoms that
uniquely originate and are carried by some regular event.

A protoskeleton $\skel_X(\rl,P,\Theta_X,\prec,N,U)$ is a
\index{preskeleton}\emph{preskeleton} if the following properties hold.
\begin{enumerate}
\item Sequence $\rl$ demonstrates that the strands in $\sdom(\Theta_X)$
  satisfy the conditions for being a part of a run of protocol~$P$.
\item Relation $\prec$ is transitive, irreflexive, and includes the
  strand succession relation $(\Rightarrow)$.
\item If $n\prec n'$, then either $n\Rightarrow n'$,
  $\evt_{\Theta_X}(n)=\outbnd t$ and   $\evt_{\Theta_X}(n')=\inbnd
  t'$, or $n\prec n''\prec n'$ for some~$n''$.
\item Each atom in $N$ is carried by no event, and each variable
  in the atom occurs in some event.
\item Each atom in $U$ is carried by some event.
\item $N$ includes the non-originating atoms inherited from roles via
  the role map.
\item $U$ includes the uniquely originating atoms inherited from roles via
  the role map.
\end{enumerate}

\begin{sloppypar}
Let $\orig_k(t)$ be the set of nodes at which~$t$ originates in
preskeleton~$k$, and~$\gain_k(t)$ be the set of nodes at which~$t$ is
gained in~$k$.  Preskeleton $\skel_X(\rl,P,\Theta_X,\prec,N,U)$ is a
\index{skeleton}\emph{skeleton} if each atom in $U$ originates on at
most one strand, and the node of origination precedes each node that
gains the atom, i.e.\@ for every~$t\in U$, $n_0\in\orig_k(t)$ and
$n_1\in\gain_k(t)$ implies~$n_0\prec n_1$.
\end{sloppypar}

Let $k_0=\skel_X(rl_0,P,\Theta_0,\prec_0,N_0,U_0)$ and
$k_1=\skel_Y(rl_1,P,\Theta_1,\prec_1,N_1,U_1)$ be preskeletons.  There
is a \index{preskeleton
  homomorphism}\index{homomorphism!preskeleton}\emph{preskeleton
  homomorphism}\label{def:preskeleton homomorphism} from~$k_0$
to~$k_1$ if~$\phi$ and~$\sigma$ are maps with the following
properties:
\begin{enumerate}
\item $\phi$ maps strands of~$k_0$ into those of~$k_1$, and nodes as
  $\phi((s,i))=(\phi(s),i)$, that is $\phi$ is in
  $\sdom(\Theta_0)\to\sdom(\Theta_1)$;
\item $\sigma\colon\algbara_X\to\algbara_Y$ is a message algebra homomorphism;
\item $n\in\nodes(\Theta_0)$ implies
  $\sigma(\evt_{\Theta_0}(n))=\evt_{\Theta_1}(\phi(n))$;
\item $n_0\prec_0
n_1$ implies $\phi(n_0)\prec_1\phi(n_1)$;
\item $\sigma(N_0)\subseteq N_1$;
\item $\sigma(U_0)\subseteq U_1$;
\item $t\in U_0$ implies
  $\phi(\orig_{k_0}(t))\subseteq\orig_{k_1}(\sigma(t))$.
\end{enumerate}

A homomorphism is \index{strandwise injective
  homomorphism}\index{homomorphism!strandwise
  injective}\emph{strandwise injective} if its strand map is
injective.  Two preskeletons are isomorphic if they are related by
\index{isomorphic preskeletons}\index{preskeletons!isomorphic}
\label{def:isomorphic preskeletons}
strandwise injective homomorphism in both directions.  A homomorphism
is \index{nodewise isomorphic
  homomorphism}\index{homomorphism!nodewise isomorphic}\emph{nodewise
  isomorphic} if the strand map~$\phi$ implies a bijection on nodes,
and $n_0\prec_1 n_1$ implies $\phi^{-1}(n_0)\prec_0\phi^{-1}(n_1)$.  A
skeleton is \index{realized
  skeleton}\index{skeleton!realized}\emph{realized} if there is a
nodewise isomorphic homomorphism from it to a skeleton that realizes a
bundle, and message component of the homomorphism is injective.

Our formalism requires that every protocol include a
listener\label{def:listeners} role\index{listener role} of the form:
$\lsn(x\colon\top)=\role(\seq{\inbnd x,\outbnd
  x},\emptyset,\emptyset)$.  Instances of this role are sometimes used
to make penetrator derived messages visible in skeletons.  We say
skeleton~$k$ \emph{realizes modulo listeners} bundle
$\Upsilon(\Theta,\to) $ if~$k$ realizes $\Upsilon(\Theta',\to')$ and
\hbox{$\Upsilon(\Theta,\to)$} is the result of removing listener
strands, and adjusting the communication ordering $\to$ appropriately.

The set of bundles denoted by preskeleton~$k$, $\sembrack{k}$, is:
$$\sembrack{k} = \{\Upsilon\mid
\mbox{$k\homomorphism{\phi,\sigma}k'$ and $k'$ realizes modulo
  listeners $\Upsilon$}\}$$ A {\cpsa} algorithm is \emph{complete} if
when given a preskeleton~$k$, either the algorithm diverges, or else it
terminates and produces a finite set of realized skeletons~$K$, such
that $\sembrack{k}=\bigcup_{k'\in K}\sembrack{k'}$.
\label{def:preskeleton denotation}

\section{Blanchet's Simple Example Protocol}\label{sec:blanchet's
  simple example protocol}

The following protocol is a simplified version of the Denning-Sacco
key distribution protocol~\cite{DenningSacco81} due to Bruno Blanchet.
$$\begin{array}{r@{{}:{}}l}
A\to B&\enc{\enc{s}{a^{-1}}}{b}\\
B\to A&\enc{d}{s}
\end{array}$$
Symmetric key~$s$ is freshly generated, asymmetric keys $a^{-1}$ and
$b^{-1}$ are uncompromised, and the goal of the protocol is to keep
data~$d$ secret.  The protocol was constructed with a known flaw for
expository purposes.

This {\cpsa} description of the protocol has an initiator and a
responder role.
$$\begin{array}{r@{{}={}}l}
\init(a,b\colon\srt{A},s\colon\srt{S}, d\colon\srt{D})&
\role(\seq{\outbnd\enc{\enc{s}{a^{-1}}}{b},\inbnd\enc{d}{s}},
\emptyset,\emptyset)\\
\resp(a,b\colon\srt{A},s\colon\srt{S}, d\colon\srt{D})&
\role(\seq{\inbnd\enc{\enc{s}{a^{-1}}}{b},\outbnd\enc{d}{s}},
\emptyset,\emptyset)
\end{array}$$
where we use~\srt{A} for sort \dom{akey}, \srt{S} for sort \dom{skey}, and
\srt{D} for sort \dom{data} to save space.  The algebra for the initiator
role is generated from $X$, where $X_{\srt{A}}=\{a,b\}$, $X_{\srt{S}}=\{s\}$,
$X_{\srt{D}}=\{d\}$, $X_{\dom{text}}=\emptyset$, $X_{\dom{name}}=\emptyset$,
and $X_\top=\emptyset$.

An interesting point of view for analysis is to see if the
authentication goals of the initiator are met.  To do so, we assume
there was full length run of an initiator strand, and let {\cpsa}
determine what else must have happened.  Let variable set $Y=a,b\colon
\srt{A},s\colon\srt{S}, d\colon\srt{D}$.  The point-of-view skeleton is:
$$\begin{array}{r@{}ll}
\skel_Y(
&\seq{\init(a_0,b_0,s_0,d_0)},
&\mbox{Role map}\\
&\{\init(a_0,b_0,s_0,d_0),\resp(a_1,b_1,s_1,d_1)\},
&\mbox{Protocol}\\
&\seq{\seq{\outbnd\enc{\enc{s}{a^{-1}}}{b},\inbnd\enc{d}{s}}},
&\mbox{Traces}\\
&\emptyset,
&\mbox{Node orderings}\\
&\{a^{-1},b^{-1}\},
&\mbox{Non-origination}\\
&\{s\})
&\mbox{Unique origination}
\end{array}$$
where the variable set that generates the algebra for the initiator
and responder roles have been renamed so as to avoid conflicts with
the variable set~$Y$ used by the skeleton.

The skeleton produced by {\cpsa} for this problem follows.  Notice
that the two strands agree on the key~$b$ used in the outermost
encryption of their first message, and indication that the
authentication goals of the initiator are met.  See
Figure~\ref{fig:blanchet's protocol} to see the structure of the shape.
$$\begin{array}{r@{}ll}
\skel_Y(
&\seq{\init(a_0,b_0,s_0,d_0),\resp(a_1,b_1,s_1,d_1)},
&\mbox{Role map}\\
&\{\init(a_0,b_0,s_0,d_0),\resp(a_1,b_1,s_1,d_1)\},
&\mbox{Protocol}\\
&\langle\begin{array}[t]{@{}l}
\seq{\outbnd\enc{\enc{s}{a^{-1}}}{b},\inbnd\enc{d}{s}},\\
\seq{\inbnd\enc{\enc{s}{a^{-1}}}{b},\outbnd\enc{d}{s}}\rangle,
\end{array}
&\mbox{Traces}\\
&\{(0,0)\prec(1,0),(1,1)\prec(0,1)\},
&\mbox{Node orderings}\\
&\{a^{-1},b^{-1}\},
&\mbox{Non-origination}\\
&\{s\})
&\mbox{Unique origination}
\end{array}$$
The homomorphism from the point-of-view skeleton to the shape is
$$(\seq{1},\{a\mapsto a, b\mapsto b, s\mapsto s, d\mapsto d\}).$$

\begin{figure}
\begin{center}
\begin{tabular}{c@{\qquad}c}
$\xymatrix{
\txt{\strut init}&\txt{\strut resp}\\
\bullet\ar@{=>}[d]\ar[r]&\bullet\ar@{=>}[d]\\
\bullet&\bullet\ar[l]}$
&
$\xymatrix{
\txt{\strut resp}&\txt{\strut init}\\
\bullet\ar@{=>}[d]&\bullet\ar@{-->}[l]\\
\bullet&}$
\end{tabular}
\end{center}
\caption{Shapes for Blanchet's Protocol}\label{fig:blanchet's protocol}
\end{figure}

An analysis of the authentication goals for the responder shows the
flaw built into the protocol.  To do the analysis, assume there was a
full length run of a responder strand, and let {\cpsa} determine what
else must have happened.  In this case, the point-of-view skeleton is:
$$\begin{array}{r@{}ll}
\skel_Y(
&\seq{\resp(a_0,b_0,s_0,d_0)},
&\mbox{Role map}\\
&\{\init(a_0,b_0,s_0,d_0),\resp(a_1,b_1,s_1,d_1)\},
&\mbox{Protocol}\\
&\seq{\seq{\inbnd\enc{\enc{s}{a^{-1}}}{b},\outbnd\enc{d}{s}}},
&\mbox{Traces}\\
&\emptyset,
&\mbox{Node orderings}\\
&\{a^{-1},b^{-1}\},
&\mbox{Non-origination}\\
&\{s\})
&\mbox{Unique origination}
\end{array}$$

The shape generated by {\cpsa} follows.  An early indication of a
problem is the variable set for the shape has three asymmetric keys.
Let variable set $Z=a,b,b'\colon\srt{A},s\colon\srt{S}, d\colon\srt{D}$.  The shape
is:
$$\begin{array}{r@{}ll}
\skel_Z(
&\seq{\resp(a_0,b_0,s_0,d_0),\init(a_1,b_1,s_1,d_1)},
&\mbox{Role map}\\
&\{\init(a_0,b_0,s_0,d_0),\resp(a_1,b_1,s_1,d_1)\},
&\mbox{Protocol}\\
&\langle\begin{array}[t]{@{}l}
\seq{\inbnd\enc{\enc{s}{a^{-1}}}{b},\outbnd\enc{d}{s}},\\
\seq{\outbnd\enc{\enc{s}{a^{-1}}}{b'}}\rangle,
\end{array}
&\begin{array}[t]{@{}l}
\mbox{Traces}\\
\mbox{\emph{Note key is $b'$ not $b$!}}
\end{array}\\
&\{(1,0)\prec(0,0)\},
&\mbox{Node orderings}\\
&\{a^{-1},b^{-1}\},
&\mbox{Non-origination}\\
&\{s\})
&\mbox{Unique origination}
\end{array}$$

Notice that the two strands do not agree on the key used in the outermost
encryption of their first message---an authentication failure.  To see
that the authentication failure leads to the failure to protect the
secrecy of data~$d$, the protocol is analyzed using the following
point-of-view:
$$\begin{array}{r@{}ll}
\skel_Y(
&\seq{\resp(a_0,b_0,s_0,d_0),\lsn(x)},
&\mbox{Role map}\\
&\{\init(a_0,b_0,s_0,d_0),\resp(a_1,b_1,s_1,d_1),\lsn(x)\},
&\mbox{Protocol}\\
&\seq{
\seq{\inbnd\enc{\enc{s}{a^{-1}}}{b},\outbnd\enc{d}{s}},
\seq{\inbnd d,\outbnd d}},
&\mbox{Traces}\\
&\{(0,1)\prec(1,0),
&\mbox{Node orderings}\\
&\{a^{-1},b^{-1}\},
&\mbox{Non-origination}\\
&\{s,d\})
&\mbox{Unique origination}
\end{array}$$

{\cpsa} finds a shape that shows how data~$d$ is revealed to the adversary.

\section{Dolev-Yao Example 1.3}

The intended run of the protocol in the Dolev-Yao Example 1.3 is:
$$\begin{array}{r@{{}:{}}l}
A\to B&\enc{\enc{m}{b},a}{b}\\
B\to A&\enc{\enc{m}{a},b}{a}
\end{array}$$
assuming text~$m$ is freshly generated, and asymmetric keys $a^{-1}$
and $b^{-1}$ are uncompromised.

The {\cpsa} description of the protocol also has an initiator and a
responder role.
$$\begin{array}{r@{{}={}}l}
\init(a,b\colon\srt{A}, m\colon\srt{D})&
\role(\seq{\outbnd\enc{\enc{m}{b},a}{b},\inbnd\enc{\enc{m}{a},b}{a}},
\emptyset,\emptyset)\\
\resp(a,b\colon\srt{A}, m\colon\top)&
\role(\seq{\inbnd\enc{\enc{m}{b},a}{b},\outbnd\enc{\enc{m}{a},b}{a}},
\emptyset,\emptyset)\\
\end{array}$$

An interesting point of view for analysis is to see if~$m$ is kept
secret after the initiator sends its message.  Let variable set
$Z=a,b\colon\srt{A}, m\colon\srt{D}$.  The initial scenario preskeleton is:
$$\begin{array}{r@{}ll}
\skel_Z(
&\seq{\init(a_0,b_0,m_0),\lsn(x)},
&\mbox{Role map}\\
&\{\init(a_0,b_0,m_0),\resp(a_1,b_1,m_1),\lsn(x)\},
&\mbox{Protocol}\\
&\seq{\seq{\outbnd\enc{\enc{m}{b},a}{b}},\seq{\inbnd m}},
&\mbox{Traces}\\
&\emptyset,
&\mbox{Node orderings}\\
&\{a^{-1},b^{-1}\},
&\mbox{Non-origination}\\
&\{m\})
&\mbox{Unique origination}
\end{array}$$
where the variable set that generates the algebra for the initiator
role has been renamed so as to avoid conflicts with the variable set~$Z$
used by the preskeleton.

\begin{figure}
$$\xymatrix{
\txt{\strut init}&\txt{\strut }&\txt{\strut resp}&\txt{\strut resp}\\
\bullet\ar[rrr]&&&\bullet\ar@{=>}[d]\\
&&\bullet\ar@{=>}[d]&\bullet\ar@{-->}[l]\\
&\bullet\ar@{=>}[d]&\bullet\ar@{-->}[l]&\\
&\bullet&&}$$
\caption{Dolev-Yao Example 1.3 Shape}\label{fig:dy shape}
\end{figure}

{\cpsa} determines $m$ is not kept secret by producing the shape in
Figure~\ref{fig:dy shape}.  The added strands in the shape are
instances of responder roles.  The strands in the shape are:
$$\begin{array}{l}
\seq{\outbnd\enc{\enc{m}{b},a}{b}}\\
\seq{\inbnd m}\\
\seq{\inbnd\enc{\enc{m}{b},a'}{b},\outbnd\enc{\enc{m}{a'},b}{a'}}\\
\seq{\inbnd\enc{\enc{\enc{m}{b}, a}{b},a''}{b},
\outbnd\enc{\enc{\enc{m}{b},a}{a''},b}{a''}}
\end{array}$$
The non-origination and unique origination assumptions are as they are
in the initial scenario preskeleton.  An interesting exercise left for
the reader is to produce a bundle that is realized by the shape.

\section{Exercise}

Consider the following roles.
$$\begin{array}{r@{{}={}}l}
\init(a,b\colon\srt{A})&\role(\seq{\outbnd(a,b),\inbnd(b,a)},
\emptyset,\emptyset)\\
\resp(a,b\colon\srt{A})&\role(\seq{\inbnd(a,b),\outbnd(b,a)},
\emptyset,\emptyset)
\end{array}$$
Let $X=x,y\colon\srt{A}$ and
$\begin{array}[t]{@{}r@{}l}
k=\skel_X(
&\seq{\init(a,b),\resp(a,b),\resp(a,b)},\\
&\{\init(a,b),\resp(a,b)\},\\
&\seq{\begin{array}[t]{@{}l}
\seq{\outbnd(x,y),\inbnd(y,x)},\\
\seq{\inbnd(x,y),\outbnd(y,x)},\\
\seq{\inbnd(x,y),\outbnd(y,x)}},
\end{array}\\
&\mbox{Node ordering in Figure~\ref{fig:exercise}},\\
&\emptyset,\\
&\emptyset)
\end{array}$

\begin{figure}
$$\xymatrix{
\txt{\strut init}&\txt{\strut resp}&\txt{\strut resp}\\
\bullet\ar@{=>}[d]\ar@/^/[rr]\ar[r]&\bullet\ar@{=>}[d]&\bullet\ar@{=>}[d]\\
\bullet&\bullet\ar[l]&\bullet\ar@/^/[ll]}$$
\caption{Exercise Skeleton}\label{fig:exercise}
\end{figure}

What is $\sembrack{k}$?

One member is shown in Figure~\ref{fig:bundle}.

\begin{figure}
$$\begin{array}{ll}
\mbox{init}&\seq{\outbnd(x,y),\inbnd(y,x)}\\
\mbox{resp}&\seq{\inbnd(x,y),\outbnd(y,x)}\\
\mbox{resp}&\seq{\inbnd(x,y),\outbnd(y,x)}\\
\mbox{pair}&\seq{\inbnd(y,x),\inbnd(y,x),\outbnd((y,x),(y,x))}\\
\mbox{sep}&\seq{\inbnd((y,x),(y,x)),\outbnd(y,x)}
\end{array}$$

$$\xymatrix{
\txt{\strut init}&\txt{\strut resp}&\txt{\strut resp}&\txt{\strut pair}&\txt{\strut sep}\\
\bullet\ar@{=>}[dddd]\ar@/^/[rr]\ar[r]&\bullet\ar@{=>}[d]&\bullet\ar@{=>}[d]&&\\
&\bullet\ar@/_/[rrd]&\bullet\ar[r]&\bullet\ar@{=>}[d]&\\
&&&\bullet\ar@{=>}[d]&\\
&&&\bullet\ar[r]&\bullet\ar@{=>}[d]\\
\bullet&&&&\bullet\ar[llll]}$$
\caption{A Bundle Realized by the Example Skeleton}\label{fig:bundle}
\end{figure}

\section{External Syntax and Instances}\label{sec:instances}

The external syntax used by {\cpsa} is a little different than what
has been described here.  In the external syntax, the trace and the
role associated with a strand is specified by an
\index{instance}\emph{instance}.  An instance is of the form
$\inst(r,h,\sigma)$, where~$r$ is a role, $h$ specifies the length of
a trace instantiated from the role, and~$\sigma$ specifies how to
instantiate the variables in the role to obtain the trace.  Thus when
$r=\role_Y(C,N,U)$, the trace associated with $\inst(r,h,\sigma)$ is
$\sigma\circ\prefix{C}{h}$.  An instance is well-formed if $1\leq
h\leq|C|$, and $\sdom(\sigma)$ is the set of variables that occur
in~$\prefix{C}{h}$.

In the external syntax, the role map and sequence of traces are
replaced by a sequence of instances.  So for preskeleton
$\skel_X(\rl,P,\Theta_X,\prec,N,U)$, the external syntax is
$\skel_X(P,I,\prec,N,U)$, where for each $s\in\sdom(\Theta_x)$,
$I(s)=\inst(r,h,\sigma)$, $r=\rl(s)$, and the trace of
$\inst(r,h,\sigma)$ is $\Theta_X(s)$.

\chapter{Algorithms as Reduction Systems}\label{chp:algorithms}

Algorithms in this paper are specified as abstract reduction
systems~\cite[Chapter~2]{BaaderNipkow98}.  A reduction system is a
pair $(A,\to)$, where reduction~$\to$ is a binary relation
$\mathord{\to}\subseteq A\times A$.  Element $x\in A$ is a
\index{normal form}\emph{normal form} if there is no~$y$ such that
$x\to y$.  The transitive closure of~$\to$ is~$\to^+$.  The reflexive
transitive closure of~$\to$ is~$\to^\ast$.  A reduction is confluent
if $x\to^\ast y_0$ and $x\to^\ast y_1$ implies there is a~$z$ such
that $y_0\to^\ast z$ and $y_1\to^\ast z$.  A reduction is terminating
if there are no infinite descending chains.  A reduction is convergent
if it is confluent and terminating.

Let $\alg{K}$ be the set of preskeletons.  Algorithms are specified as
reduction systems of the form $(\alg{K},\to)$, which are then used to
specify a related setwise reduction system of the form
$(\pow{\alg{K}},\mathord\twoheadrightarrow)$.  Setwise reduction
systems are the ones with the interesting normal forms and confluence
properties.  In a setwise reduction system, reduction rewrites one
element of a set to a set of elements.

\begin{defn}[Setwise Reduction System]
The \emph{setwise reduction system}\index{setwise reduction system} of
binary relation $\rightsquigarrow\subseteq\alg{K}\times\pow{\alg{K}}$
is a reduction system $(\pow{\alg{K}},\twoheadrightarrow)$, where for
each $K_0\in\pow{\alg{K}}$, $K_0\twoheadrightarrow K_1$ if for some
$k_0\in K_0$, $k_0\rightsquigarrow K_2$, $K_1=K_2\cup
(K_0\setminus\{k_0\})$, and $K_1\neq K_0$.
\end{defn}

The {\cpsa} algorithm will be specified as a setwise term reduction
system, where the initial problem is given a singleton
in~$\pow{\alg{K}}$, and the answers computed by an implementation of
the algorithm are a normal form of the setwise reduction
relation~$\twoheadrightarrow_k$ defined in Chapter~\ref{chp:skeleton
  reduction}.

In what follows the relation $k\rightsquigarrow K$ is defined in terms
of $\to\subseteq\alg{K}\times\alg{K}$ by specifying
$\{k\}\twoheadrightarrow K$ using~$\to$, so the $\rightsquigarrow$
relation is not explicitly defined.

We regard sets of preskeletons as factored by isomorphism, where each
set has at most one representative of the equivalence class of
isomorphic preskeletons.  The definition of isomorphic preskeletons
is given on Page~\pageref{def:isomorphic preskeletons}.

The {\cpsa} Design~\cite{cpsadesign09} describes an extension of a
message algebra signature that models the data structures used in the
{\cpsa} program.  The terms over the extended signature include ones
that model preskeletons.  Sets of terms of sort preskeleton are the
domain of our setwise reduction systems.

\chapter{Primitive Preskeleton Operators}\label{chp:prim preskel ops}

The are four primitive operators\index{operators} on preskeletons used
by {\cpsa} to solve authentication tests.  Each operator is a partial
map from preskeletons to preskeletons.

\begin{defn}[Substitution Operator]\label{def:substitution operator}
For order-sorted substitution~$\sigma\colon X\to\alg{A}_Y$,
the operator~$\ops{S}_\sigma$ is:
$$\begin{array}{l}
\ops{S}_\sigma(\skel_X(\rl,P,\Theta_X,\prec,N,U)) = {}\\
\quad\skel_Y(\rl,P,
s\mapsto \comp\sigma{\Theta_X(s)},\prec,\sigma(N), \sigma(U))
\end{array}$$
\end{defn}

For $k'=\ops{S}_\sigma(k)$, there is a homomorphism from~$k$ to~$k'$
only if for all $t\in U_k$,
$\orig_k(t)\subseteq\orig_{k'}(\sigma(t))$.  The structure preserving
maps associated with the homomorphism are~$\idphi$ and~$\sigma$.

\begin{defn}[Compression Operator]\label{def:compression operator}
For distinct strands~$s$ and~$s'$, operator~$\ops{C}_{s,s'}$
compresses strand~$s$ into~$s'$.
$$\begin{array}{l}
\ops{C}_{s,s'}(\skel_X(\rl,P,\Theta_X,\prec,N,U)) = {}\\
\quad \skel_X(\comp\rl{\phi'_s},P,
\comp{\Theta_X}{\phi'_s},\prec',N,U)
\end{array}$$
where
$$\begin{array}{r@{}c@{}l}
\phi'_s(j)&{}={}&\left\{
\begin{array}{ll}
j+1&\mbox{if $j\geq s$}\\
j&\mbox{otherwise,}
\end{array}\right.
\end{array}$$
relation $\prec'$ is the transitive closure of $\phi_{s,s'}(\prec)$, and
$$\begin{array}{r@{}c@{}l}
\phi_{s,s'}(j)&{}={}&\left\{
\begin{array}{ll}
\phi_s(s')&\mbox{if $j=s$}\\
\phi_s(j)&\mbox{otherwise}
\end{array}\right.\\
\phi_s(j)&{}={}&\left\{
\begin{array}{ll}
j-1&\mbox{if $j>s$}\\
j&\mbox{otherwise.}
\end{array}\right.
\end{array}$$
\end{defn}

The compression operator is only used when $\Theta_X(s)$ is a prefix
of $\Theta_X(s')$, and when there is a homomorphism from~$k$ to
$\ops{C}_{s,s'}(k)$.  The structure preserving maps associated with
the homomorphism are~$\phi_{s,s'}$ and~$\idsigma$.  Note that the
compression operator is defined only when relation~$\prec'$ is
asymmetric, and that $\comp{\phi_{s,s'}}{\phi'_s}=\idphi$.

\begin{defn}[Ordering Enrichment Operator]\label{def:order enrichment operator}
Operator~$\ops{O}$ applied to~$k$ enriches~$\kprec{k}$ by adding all
elements implied by unique origination.  That is $\ops{O}(k)=
\skel_X(\rl,P,\Theta_X,\prec',N,U))$, where
$k=\skel_X(\rl,P,\Theta_X,\prec,N,U)$ and $\prec'=
(\prec\cup\{(n_0,n_1)\mid n_0\in\orig(k,t)\land
n_1\in\gain(k,t)\})^\ast$.
\end{defn}

The ordering enrichment operator is total and idempotent.  The
structure preserving maps associated with the operator's homomorphism
are~$\idphi$ and~$\idsigma$, i.e.\ the homomorphism is an embedding.

\begin{defn}[Augmentation Operator]\label{def:augmentation operator}
For node~$n$, role~$r$, and trace~$C$, operator~$\ops{A}_{n,r,C}$ is:
$$\begin{array}{l}
\ops{A}_{n,r,C}(\skel_X(\rl,P,\Theta_X,\prec,N,U)) = {}\\
\quad\skel_{X'}(\rl\append r,P,
\Theta_X(s)\append C,\prec',N',U')
\end{array}$$
where $X'$ is $X$ extended to include the variables in~$C$,
$\prec'$ is the minimal extension of $\prec$ such that
$(|\Theta_X|+1,|C|)\prec' n$, $N'$ is $N$ extended with
non-origination assumptions inherited from~$r$ by~$C$, and likewise
for~$U'$.
\end{defn}

The structure preserving maps associated with the augmentation
operator's homomorphism are~$\idphi$ and~$\idsigma$, i.e.\ the
homomorphism is an embedding.

\chapter{Reductions on Preskeletons}\label{chp:preskeleton reduction}

This chapter describes the algorithm used to transform a preskeleton
into a skeleton as a setwise term reduction system
$(\alg{K},\twoheadrightarrow)$.  Recall that the relation
$k\rightsquigarrow K$ in Chapter~\ref{chp:algorithms} is defined in
terms of $\to\subseteq\alg{K}\times\alg{K}$ by specifying
$\{k\}\twoheadrightarrow K$ using~$\to$.  Additionally, when
$k=\skel(\rl,P,\Theta,\prec,N,U)$, $U_k$ is used to name~$U$, and so
forth for other components of~$\skel$.

If a preskeleton $k$ is not a skeleton, then it is either because some
$t\in U_k$ actually originates at more than one node, or because for
some $t\in U_k$, there is a node $n_1\in\gain_k(t)$, and a node
$n_0\in\orig_k(t)$ such that $n_0\nprec_k n_1$.  A preskeleton in
which an atom assumed to be uniquely originating originates more than
once is simply expunged.  The second obstruction is resolved by
enriching node orderings.  A \index{hulled
  preskeleton}\index{preskeleton!hulled}\emph{hulled preskeleton} is a
preskeleton in which every uniquely originating atom originates at
most once, but it may lack some node orderings needed to be a
skeleton.

Skeletons may contain ``effectively equivalent'' strands.  After
converting preskeletons to skeletons, a preskeleton reduction system
may remove effectively equivalent strands using a process called
\index{thinning}thinning.  A skeleton without any effectively
equivalent strands is called a \index{thinned
  skeleton}\index{skeleton!thinned}\emph{thinned skeleton}.

\begin{defn}[Hulling Reduction]\label{def:hulling}
A non-hulled preskeleton~$k$ is expunged.  The setwise hulling
reduction rule is $\{k\}\setreduction{\ops{H}}\{\}$ when~$k$ is not
hulled.
\end{defn}

\begin{defn}[Order Enrichment]
Suppose hulled preskeleton~$k_0$ is not a skeleton.  Hulled
preskeleton~$k_0$ reduces to skeleton~$k_1$ by order enrichment, written
$k_0\reduction{\ops{O}}k_1$, iff~$k_1$ is the result of adding node
orderings implied by origination.  That is,
$\kprec{k_1}= (\kprec{k_0}\cup\{(n_0,n_1)\mid n_0\in\orig(k_0,t)\land
n_1\in\gain(k_0,t)\})^\ast$.
There is a homomorphism from~$k_0$ to~$k_1$ that is an embedding.  For
the setwise order enrichment reduction,
$\{k_0\}\setreduction{\ops{O}}\{k_1\mid k_0\reduction{\ops{O}}k_1\}$
when $k_0$ is a hulled preskeleton that is not a skeleton.
\end{defn}

The thinning operator~$\ops{T}_s$ removes an \index{effectively
  equivalent strands}effectively equivalent strand.

\begin{defn}[Thinning]\label{def:thinning}
Two strands $s$ and $s'$ in a skeleton $k$ are \emph{effectively
  equivalent} if $k$ with $s$ removed and $k$ with $s'$ removed are
isomorphic.  (Page~\pageref{def:isomorphic preskeletons} defines
isomorphic skeletons.)  Skeleton $k_0$ reduces to skeleton $k_1$ by
thinning, written $k_0\reduction{\ops{T}_s}k_1$, if and only if there
exists a strand $s'$ that is effectively equivalent to $s$ in $k_0$
and $k_1$ is isomorphic to both $k_0$ with $s$ removed and $k_0$ with
$s'$ removed.  In detail, $k_0\reduction{\ops{T}_s}k_1$ if and only if
there exists a strand $s'$ such that
\begin{enumerate}
\item strands $s$ and $s'$ are not in the image of~$k_0$'s
  strand map from the point-of-view skeleton;
\item $\Theta_{k_0}(s)$ is isomorphic to $\Theta_{k_0}(s')$;
\item let $k_1$ be $k_0$ with strand~$s$ removed, along with
  extraneous origination assumptions;
\item let $k_2$ be $k_0$ with strand~$s'$ removed, along with
  extraneous origination assumptions;
\item $k_1$ is isomorphic to $k_2$.
\end{enumerate}

For the setwise thinning reduction,
$\{k_0\}\setreduction{\ops{T}_s}\{k_1\mid
k_0\reduction{\ops{T}_s}k_1\}$, when there is a $k_1$ such that
$k_0\reduction{\ops{T}_s}k_1$.
\end{defn}

Thinning in {\cpsa} is also implemented for the case of sets of
multiple strands rather than pairs of individual strands.  The
correctness of thinning has yet to be demonstrated.

\section{Reduction Systems}

Notice that a setwise hulling reduction may produce the empty set, but
a setwise order enrichment and thinning reduction never does.

There are two preskeleton reduction systems, one with thinning, and one
without.  For the one without, let reduction $\twoheadrightarrow=
\bigcup_{s,s'}\setreduction{\ops{H}}\cup\setreduction{\ops{O}}$.

\begin{defn}[Preskeleton Reduction System]
Preskeleton~$k_0$ reduces to skeleton~$k_1$,
written~$k_0\reduction{\fn{skel}}k_1$\index{skel@\fn{skel}}, if
$\{k_0\}\twoheadrightarrow^\ast K$, $k_1\in K$, and~$K$ is a normal
form of~$\twoheadrightarrow$.
\end{defn}

For each skeleton~$k$, $k\reduction{\fn{skel}}k$.

When using thinning, let reduction $\twoheadrightarrow=
\bigcup_{s,s'}\setreduction{\ops{H}}\cup\setreduction{\ops{O}}\cup
\bigcup_s\setreduction{\ops{T}_s}$.

\begin{defn}[Preskeleton Reduction System with Thinning]
Preskeleton~$k_0$ reduces to thinned skeleton~$k_1$,
written~$k_0\reduction{\fn{tskel}}k_1$\index{tskel@\fn{tskel}}, if
$\{k_0\}\twoheadrightarrow^\ast K$, $k_1\in K$, and~$K$ is a normal
form of~$\twoheadrightarrow$.
\end{defn}

For each thinned skeleton~$k$, $k\reduction{\fn{tskel}}k$.

\chapter{Penetrator Derivable}\label{chp:penetrator derivable}

For each algebra, the powers of the adversary are defined by a set of
roles.  For the Basic Crypto Signature in Figure~\ref{fig:bcs}, the
traces of the penetrator roles are in Figure~\ref{fig:pen traces}.
Penetrator roles make no origination assumptions.

\begin{figure}
\[\begin{array}{r@{{}={}}l}
\fn{base}(t)&\seq{\outbnd t},\mbox{ where $t$ is an atom}\\
\fn{tag}(t)&\seq{\outbnd t},\mbox{ where $t$ is a tag}\\
\fn{pair}(t_0,t_1)&\seq{\inbnd t_0,\inbnd t_1, \outbnd(t_0,
  t_1)}\\
\fn{sep}(t_0,t_1)&\seq{\inbnd(t_0, t_1),\outbnd
    t_0,\outbnd t_1}\\
\fn{enc}(t_0,t_1)&\seq{\inbnd t_0,\inbnd t_1,
\outbnd\enc{t_0}{t_1}}\\
\fn{dec}(t_0,t_1)&\seq{\inbnd\enc{t_0}{t_1},
\inbnd t_2,\outbnd t_0},\mbox{ where $t_2=\fn{inv}(t_1)$}\\
\fn{hash}(t)&\seq{\inbnd t,\outbnd\hash t}\\
\fn{put}(c,t)&\seq{\inbnd t, \outbnd[c,t]}\\
\fn{take}(c,t)&\seq{\inbnd[c,t], \outbnd t}
\end{array}\]
\caption{Penetrator Traces}\label{fig:pen traces}
\end{figure}

The context in which penetrator strands appear determine the messages
the adversary can derive.  The context includes previously sent
messages and atoms it is forbidden to originate.  An atom that is
assumed to be non-originating must be avoided as is a uniquely
originating atom that is assumed to originate on a regular strand.

The ternary relation \der{t}{T_p}{T_a} states that message~$t$ is
penetrator derivable from previously sent messages~$T_p$ while
avoiding atoms~$T_a$.  The relation is defined by a set of inference
rules.  Most of the rules are justified by a penetrator role that when
instantiated, derives a message in the conclusion of the rule.  Note
there are no rules that pertain to channel messages.  Channel messages
are handled elsewhere.

The first rule states that no additional penetrator behavior is required
to derive~$t$ if it has been previously sent.
\begin{equation}
  \infer{t\in{T_p}}{\der{t}{T_p}{T_a}}
\end{equation}
%
By $\fn{base}(t)$:
\begin{equation}\label{eqn:base}
  \infer{\mbox{$t$ an atom}\qquad t\notin T_a}{\der{t}{T_p}{T_a}}
\end{equation}
%
By $\fn{tag}(\mbox{``\ldots''})$:
\begin{equation}\label{eqn:tag}
  \der{\mbox{``\ldots''}}{T_p}{t_a}
\end{equation}
%
By $\fn{pair}(t_0,t_1)$:
\begin{equation}\label{eqn:pair}
  \infer{\der{t_0}{T_p}{T_a}\qquad\der{t_1}{T_p}{T_a}}
        {\der{(t_0,t_1)}{T_p}{T_a}}
\end{equation}
%
By $\fn{sep}(t_0,t_1)$:
\begin{equation}\label{eqn:sep0}
  \infer{\der{(t_0,t_1)}{T_p}{T_a}}{\der{t_0}{T_p}{T_a}}
\end{equation}
%
By $\fn{sep}(t_0,t_1)$:
\begin{equation}\label{eqn:sep1}
  \infer{\der{(t_0,t_1)}{T_p}{T_a}}{\der{t_1}{T_p}{T_a}}
\end{equation}
%
By $\fn{enc}(t,k)$:
\begin{equation}\label{eqn:enc}
  \infer{\der{t}{T_p}{T_a}\qquad\der{k}{T_p}{T_a}}{\der{\enc{t}{k}}{T_p}{T_a}}
\end{equation}
%
By $\fn{dec}(t,k)$:
\begin{equation}\label{eqn:dec}
  \infer{\der{\enc{t}{k}}{T_p}{T_a}\qquad\der{\fn{inv}{k}}{T_p}{T_a}}{\der{t}{T_p}{T_a}}
\end{equation}
%
By $\fn{hash}(t)$:
\begin{equation}\label{eqn:hash}
  \infer{\der{t}{T_p}{T_a}\qquad\der{k}{T_p}{T_a}}{\der{\hash{t}}{T_p}{T_a}}
\end{equation}
%
%% By $\fn{put}(c,t)$:
%% \begin{equation}\label{eqn:put}
%%   \infer{\der{t}{T_p}{T_a}\qquad\mbox{$c$ not authenticated}}%
%%         {\der{[c,t]}{T_p}{T_a}}
%% \end{equation}
%% %
%% By $\fn{take}(c,t)$:
%% \begin{equation}\label{eqn:take}
%%   \infer{\der{[c,t]}{T_p}{T_a}\qquad\mbox{$c$ not confidential}}%
%%         {\der{t}{T_p}{T_a}}
%% \end{equation}
%% %
A non-base sorted variable is derivable in a bundle that instantiates
it with any message other than an element of $X_\top$.
\begin{equation}\label{eqn:top}
  \infer{t\in X_\top}{\der{t}{T_p}{T_a}}
\end{equation}

\begin{defn}[Sent Channel Messages]
The \index{sent channel messages}\emph{sent channel messages} of
skeleton~$k$ at~$n$ is $\sntchmsg(k,n)=\{\msg_k(n_0)\mid n_0 \kprec{k}
n, n_0\mbox{~is transmitting}\}$.
\end{defn}

\begin{defn}[Public Messages]
The \index{public messages}\emph{public messages} of
skeleton~$k$ at~$n$ is \[\pubmsg(k,n)=\{t\mid [t]\in M\}\cup
\{t\mid[c,t]\in M\land c\notin \conf(k)\},\] where $M=\sntchmsg(k, n)$,
and $\conf(k)$ is the set of channels assumed to be confidential in~$k$.

\end{defn}

\begin{defn}[Avoidance Set]\label{def:avoid}
The \index{avoidance set}\emph{avoidance set} of skeleton~$k$ is
$\avoid(k)=N_k\cup\{t\mid t\in U_k\land|\orig_k(t)|=1\}$.
\end{defn}

An atom in $\avoid(k)$ is not available to the penetrator, except
if it is exposed by a messages transmission.  Clearly, only uniquely
originating atoms can be exposed.

\begin{defn}[Derivable Before]\label{def:der}
A channel message~$m$ is \index{derivable before}\emph{derivable
  before} reception node~$n$ in skeleton~$k$, written
$\fn{der}(k,n,t)$, if
\begin{enumerate}
\item $m\in\sntchmsg(k,n)$,
\item $m=[t]$ and $\der{t}{T_p}{T_a}$, or
\item $m=[c,t]$, $c\notin\auth{k}$, and $\der{t}{T_p}{T_a}$,
\end{enumerate}
where $T_p=\pubmsg(k,n)$, $T_a=\avoid(k)$, and $\auth(k)$ is the set
of channels assumed to be authenticated in~$k$.
\end{defn}

\begin{defn}[Realized Node]
A reception node~$n$ is \emph{realized} in skeleton~$k$ if
$\msg_k(n)$ is derivable before~$n$ in~$k$.
\end{defn}

Notice that one can read off penetrator behavior from the proof tree
used to demonstrate that $\msg_k(n)$ is derivable before~$n$
in~$k$.  For example, if a decryption step is required by the proof,
an instance of the penetrator's decryption role is indicated.  In a
bundle, for a non-base sorted variable, there is a substitution that
maps the variable to a message that is not a non-base sorted variable.
The substitution determines the penetrator behavior associated with
the variable.

\begin{conj}[Realized Skeleton]
A skeleton is realized if and only if all of its reception nodes are
realized.
\end{conj}

\begin{proof}[Partial Proof]
Given a skeleton~$k$ in which all of its reception nodes are realized,
the combination of the regular behavior in the skeleton, the
penetrator behavior specified by the proof trees used to demonstrate
each node is realized, and a substitution for non-base sorted
variables determines a bundle.  The skeleton of the bundle may have
more non-originating atoms than is in $N_k$, however since the
extra non-originating atoms are derivable by the bundle that
realizes~$k$, the proof trees for those atoms specify any additional
penetrator behavior required.

The ``only if'' part of this proof has yet to be completed.
\end{proof}

\section{Implementation}\label{sec:derivable implementation}

The derivable before a node predicate is implemented using auxiliary
functions.

\begin{defn}[Buildable]
Message~$t$ is \index{buildable}\emph{buildable} from previously sent
messages~$T_p$ while avoiding~$T_a$, written $\fn{bld}(t,T_p,T_a)$, if
\der{t}{T_p}{T_a} without the use of Inference Rules \ref{eqn:sep0},
\ref{eqn:sep1}, and~\ref{eqn:dec}.
\end{defn}

Consider the following reduction system based on Inference Rules
\ref{eqn:sep0}, \ref{eqn:top}, \ref{eqn:sep1}, and~\ref{eqn:dec}.
$$\begin{array}{r@{}c@{}ll}
\dctx{T_p}{T_a}&{}\to{}&\dctx{T_p}{T_a\setminus\{t\}}
&\mbox{if $t\in T_p$ and $t\in T_a$}\\
\dctx{T_p}{T_a}&{}\to{}&\dctx{T_p\setminus\{x\}}{T_a}
&\mbox{if $x\in T_p$ and $x\in X_{\top}$}\\
\dctx{\{(t_0,t_1)\}\cup T_p}{T_a}&{}\to{}
&\dctx{\{t_0,t_1,(t_0,t_1)\}\cup T_p}{T_a}
&\mbox{if $t_0$ or $t_1\notin T_p$}\\
\dctx{\{\enc{t_0}{t_1}\}\cup
T_p}{T_a}&{}\to{}&\dctx{\{t_0,\enc{t_0}{t_1}\}\cup T_p}{T_a}
&\mbox{if $t_0\notin T_p$ and}\\
&&&\fn{bld}(\fn{inv}(t_1),T_p,T_a)
\end{array}$$

\begin{defn}[Decompose]\label{def:decompose}
Previously sent messages~$T_p$ and avoidance set~$T_a$
\emph{decompose} to $T'_p,T'_a$, written
$\fn{decompose}(T_p,T_a)=(T'_p,T'_a)$, if
$\dctx{T_p}{T_a}\to^\ast\dctx{T'_p}{T'_a}$
and $(T'_p,T'_a)$ is a normal form of reduction~$\to$.
\end{defn}

The penetrator derivable predicate $\der{t}{T_p}{T_a}$ is implemented as
\begin{center}
\begin{tabular}{l}
$\der{t}{T_p}{T_a}={}$\\
\quad\textbf{let} $T'_p, T'_a=\fn{decompose}(T_p,T_a)$ \textbf{in}\\
\quad$\fn{bld}(t,T'_p,T'_a)$
\end{tabular}
\end{center}

The decomposition at a node function is
\begin{center}
\begin{tabular}{l}
$\fn{dcmp}(k,n)={}$\\
\quad$\fn{decompose}(\pubmsg(k,n),\avoid(k))$
\end{tabular}
\end{center}

The derivable before a node predicate is implemented as
\begin{center}
\begin{tabular}{l}
$\fn{der}(k,n,t)={}$\\
\quad\textbf{let} $T_p, T_a=\fn{dcmp}(k,n)$ \textbf{in}\\
\quad$\fn{bld}(t,T_p,T_a)$
\end{tabular}
\end{center}

\chapter{Carried Only Within}\label{chp:cow}

A set of encryptions~$T_e$ protects critical message~$t$ in
message~$t'$ if~$t$ is carried by~$t'$ only within a member of~$T_e$.
The definition of the carried only within ({\cow}) relation to follow makes
this concept precise.  The concept is used when solving authentication
tests (Chapter~\ref{chp:auth tests}).

\begin{defn}[Ancestors]
Let $t'=t\termat p$.  The \index{ancestors}\emph{ancestors} of~$t'$
in~$t$ at~$p$ is the set $\fn{anc}(t,p)=\{t\termat p'\mid \mbox{$p'$ a
  proper prefix of~$p$}\}$.
\end{defn}

\begin{defn}[Carried Only Within]\label{def:cow}
Message~$t$ is \index{carried only within}\emph{carried only within}
set~$T_e$ in~$t'$ if for all \index{carried positions}carried
positions~$p$ of~$t$ in~$t'$, there exists an
ancestor~$t_a\in\fn{anc}(t',p)$ and $t_e\in T_e$ such that~$t_a\equiv
t_e$.
\end{defn}

The function defines $\fn{carpos}(t,t')$ is defined on
Page~\pageref{def:carried positions}, the set of positions at
which~$t'$ carries~$t$.  The interface to each algebra exports
\index{unify}$\fn{unify}_0$, where
$$\fn{unify}_0(t,t',\sigma)=\{\sigma'\circ\sigma\mid
\sigma'\in\fn{unify}(\sigma(t),\sigma(t'))\}.$$
Unification is extended to channel messages as:
\[\begin{array}{l}
\fn{unify_0}([t],[t'],\sigma)=\fn{unify_0}(t,t',\sigma)\\
\fn{unify_0}([c,t],[c',t'],\sigma)=\\
\qquad\{\sigma''\mid\sigma'\in\fn{unify_0}(c,c',\sigma)\land
\sigma''\in\fn{unify_0}(t,t',\sigma')\}
\end{array}\]

The details of a reduction on skeletons called a augmentation will be
described in Section~\ref{sec:augmentation}.  In simplified form, for
an augmentation, given~$t$,~$T_e$, and~$t'$, one must find all most
general unifiers~$\sigma$ such that~$\sigma(t)$ is carried only within
set~$\sigma(T_e)$ in~$\sigma(t')$.

\begin{figure}
\begin{center}
\begin{tabular}{l}
$\fn{cows}(t,T,t') ={}$\\
\quad $\fn{cows}_0(t,T,t',\idsigma)$
\quad --- $\idsigma$ is the identity
subst \\
\\
$\fn{cows}_0(t,T,t',\sigma) ={}$\\
\quad \textbf{if} $\sigma(t)$ is {\cow} $\sigma(T)$ at
$\sigma(t')$ \textbf{then}\\
\qquad $\{\sigma\}$\\
\quad \textbf{else}\\
\qquad\textbf{let}
$S=\fn{fold}(t,T,t',\sigma)$
\textbf{in}\\
\qquad$\bigcup_{\sigma'\in S}
\fn{cows}_0(t, T, t',\sigma')$\\
\\
$\fn{fold}(t,T,t',\sigma)={}$\\
$\quad\{\sigma'\circ\sigma\mid\sigma'\in
\fn{fold}_0(\sigma(T),\sigma(t'),\{\idsigma\},
\fn{carpos}(\sigma(t),\sigma(t')))\}$\\
\\
$\fn{fold}_0(T,t',S,\{\})=S$\\
$\fn{fold}_0(T,t',S,\seq{p}\append P)={}$\\
\quad $\fn{fold}_0(T,t',\fn{solve}(\fn{anc}(t',p),T,S),P)$\\
\\
$\fn{solve}(T,T',S)={}$\\
\quad$\{\sigma'\mid t\in T, t'\in T',
\sigma\in S, \sigma'\in\fn{unify}_0(t,t',\sigma)\}$
\end{tabular}
\end{center}
\caption{The \fn{cows} Function}\label{fig:cows}\index{cows@\fn{cows}}
\end{figure}

A carried only within solution cannot be directly computed using
Definition~\ref{def:cow}.  Given terms~$t_a$ and~$t_e$, the
$\fn{unify}_0$ function finds substitutions~$\sigma$ such
$\sigma(t_a)\equiv \sigma(t_e)$, however, the carried positions
$\fn{carpos}(\sigma(t),\sigma(t'))$, are used before the
$\fn{unify}_0$ function computes the substitution~$\sigma$.
Figure~\ref{fig:cows} displays the iterative procedure that breaks the
cyclic dependencies.  Each step of the iteration improves an
approximation of a solution to the problem.  The correctness of this
function is shown in~\cite{algimpl11}, although for a version of the
algorithm with the alternate definition for the function~$\fn{fold}$ in
Figure~\ref{fig:fold}.

\begin{figure}
\begin{center}
\begin{tabular}{l}
$\fn{fold}(t,T,t',\sigma)={}$\\
$\quad\fn{fold}_0(t,T,t',\sigma,\{\idsigma\},\fn{carpos}(\sigma(t), \sigma(t')))$\\
\\
$\fn{fold}_0(t,T,t',\sigma,S,\{\})=\{\sigma' \circ \sigma | \sigma' \in S\}$\\
$\fn{fold}_0(t,T,t',\sigma,S,\seq{p} \append P)={}$\\
\quad
$\fn{fold}_0(t,T,t',\sigma,\fn{solve}(\fn{anc}(\sigma(t'),p),\sigma(T),S),P)$\\
\end{tabular}
\end{center}
\caption{The Alternate \fn{fold} Function}\label{fig:fold}
\end{figure}

\chapter{Solving Authentication Tests}\label{chp:auth tests}

\begin{defn}[Protectors]
The \index{protectors}protectors of~$t$ are encryptions that carry~$t$
and cannot be decrypted.  Function $\fn{prot}(T_p,T_a,t)$ is undefined
if $\fn{bld}(t,T_p,T_a)$, otherwise
\[\fn{prot}(T_p,T_a,t)=\{\enc{t_0}{t_1}\in T_p\mid
\mbox{$t_0$ carries $t$ and }\lnot\fn{bld}(\fn{inv}(t_1),T_p,T_a)\}\]
\end{defn}

\begin{defn}[Escape Set]
The \index{escape set}\emph{escape set} for message~$t_c$ at~$n$ in
skeleton~$k$, $\fn{esc}(k,n,t_c)$, contains encryptions and channel
messages.  Function $\fn{esc}(k,n,t_c)=\fn{prot}(t_c,T_p,T_a)\cup M$,
where $(T_p,T_a)=\fn{dcmp}(k,n)$ (See Definition~\ref{def:decompose}),
and $M$ is the set of sent channel messages with confidential channels
that carry~$t_c$.  It is undefined when~$t_c$ is derivable in~$k$
at~$n$.
\end{defn}

\begin{defn}[Critical Position, Critical Message]\label{def:critical position}
  Position~$p$ is a \index{critical position}\emph{critical position}
  at $n$ in $k$ if for $m=\msg_k(n)$
\begin{enumerate}
\item $p$ is a carried position in~$m$,
\item $m\termat p$ is not derivable before $n$ in $\skel$,
\item either
  \begin{enumerate}
  \item $m\termat p\in U_k$,
  \item $m\termat p=\hash t_1$,
  \item $m\termat p=\enc{t_0}{t_1}$ and~$t_1$ is not derivable
    before~$n$ in~$k$, or
  \item $m=[c,t]$, $p=\seq{}$, and $c\in\auth(k)$.
  \end{enumerate}
  \label{item:nonce, encryption, or channel}
\item $\fn{anc}(t,p)\cap \fn{esc}(k,n,t\termat p)=\emptyset$.
\end{enumerate}
The message at a critical position is called a \emph{critical message.}
\end{defn}

\begin{defn}[Authentication Test]\label{def:authentication test}
  The triple $(k,n,p)$ is an \index{authentication
    test}\emph{authentication test} iff $p$ is a critical position at
  reception node $n$ in $k$.
\end{defn}

An authentication test is a \index{nonce test}\emph{nonce test} if the
critical message is an atom, it is a \index{channel test}\emph{channel
  test} if the critical message is a channel message, otherwise it is
an \index{encryption test}\emph{encryption test}.  (A hash\index{hash}
is treated as a kind of encryption in which the term that is hashed is
the encryption key.)  Observe that every critical message at a node in
a skeleton is not derivable at the node.

\begin{conj}
A reception node is unrealized iff it has a critical position.
\end{conj}

\begin{defn}[Target Messages]
Let $T_e$ be a set of messages, and~$t_c$ be a message.  The set of
\index{target messages}\emph{target messages} is
$$\fn{targ}(t_c,T_e)=\{t_c\}\cup\{t_t\mid t_e\in T_e,p\in\fn{carpos}(t_c, t_e),
t_t\in\fn{anc}(t_e,p)\}\setminus T_e.$$
\end{defn}

\begin{defn}[Critical Position Solved]\label{def:critical position solved}
Suppose~$p$ is a critical position at~$n$ in~$k_0$ and
$k_0\homomorphism{\phi,\sigma}k_1$.  Let~$t=\msg_{k_0}(n)\termat p$,
 and $T=\fn{esc}(k_0,n,t)$.  Critical position~$p$ is
\index{solved critical position}\emph{solved} in~$k_1$ after~$k_0$
at~$n$ if:
\begin{enumerate}
\item\label{item:anc}
  $\fn{anc}(\msg_{k_1}(\phi(n)),p)\cap\sigma(T)\neq\emptyset$, or
\item\label{item:aug} for some~$t_p\in\sntchmsg(k_1,\phi(n))$,
  $\sigma(t)$ is not carried only within~$\sigma(T)$ in~$t_p$, or
\item\label{item:targs}
  $\fn{targ}(\sigma(t),\sigma(T))\setminus\sigma(\fn{targ}(t,T))\neq\emptyset$
  and there are variables in $k$'s protocol that are not atoms, or
\item\label{item:dec} the decryption key of a member of~$\sigma(T)$ is
  derivable before~$\phi(n)$ in~$k_1$, or
\item\label{item:enc} $\sigma(t)$ is an encryption and its encryption
  key is derivable before~$\phi(n)$ in~$k_1$.
\end{enumerate}
\end{defn}

\section{Test Solving Steps}

A step used to solve a test is a contraction
(Definition~\ref{def:contraction}), a regular augmentation
(Definition~\ref{def:regular augmentation}), a displacement
(Definition~\ref{def:displacement}), or a listener augmentation
(Definition~\ref{def:listener augmentation}).

\begin{defn}[Contraction]\label{def:contraction}
Let~$p$ be a critical position
at~$n$ in~$k$, $t=\msg_k(n)$, and $T_e= \fn{esc}(k,n,t\termat p)$.
Suppose there is a substitution~$\sigma$ such that for
some~$t_a\in\fn{anc}(t,p)$, $t_e\in T_e$, $\sigma(t_a)=\sigma(t_e)$.
Skeleton~$k_1$ is a \index{contraction}\emph{contraction} if
$k\reduction{\ops{S}_\sigma}k_0\reduction{\fn{tskel}}k_1$.
\end{defn}

{\cpsa} computes a set of substitutions for each critical position, and
then removes some substitutions to form a complete set of most
general unifiers.  Only most general unifiers are used for
contractions.

\begin{defn}[Regular Augmentation]\label{def:regular augmentation}
Suppose substitution~$\sigma$, non-listener role~$r$, and trace~$C$
are selected as described in Section~\ref{sec:augmentation}.
Skeleton~$k_2$ is a \index{regular augmentation}\emph{regular
  augmentation} if
$k\reduction{\ops{S}_\sigma}k_0\reduction{\ops{A}_{n,r,C}}k_1
\reduction{\fn{tskel}}k_2$.
\end{defn}

\begin{defn}[Displacement]\label{def:displacement}
Let substitution~$\sigma$, non-listener role~$r$, and trace~$C$ be
selected as described in Section~\ref{sec:augmentation}, and there be
a preskeleton~$k_1$ such that
$k\reduction{\ops{S}_\sigma}k_0\reduction{\ops{A}_{n,r,C}}k_1$.
Suppose there are strands~$s$ and~$s'$, where one of them is the newly
created strand, and a most general unifier~$\sigma'$ such that
$\sigma'(\Theta_{k_1}(s)(j))\equiv \sigma'(\Theta_{k_1}(s')(j))$ for
$0\leq j<|\Theta_{k_1}(s)|$.  Skeleton~$k_4$ is a
\index{displacement}\emph{displacement} if
$k_1\reduction{\ops{S}_{\sigma'}}k_2\reduction{\ops{C}_{s,s'}}k_3
\reduction{\fn{tskel}}k_4$.
\end{defn}

\begin{defn}[Listener Augmentation]\label{def:listener augmentation}
Let~$p$ be a critical position at~$n$ in~$k$, $t_c=\msg_k(n)\termat
p$, and $T_e= \fn{esc}(k,n,t_c)$.  For each $\enc{t_0}{t_1}\in T_e$,
skeleton~$k_1$ is a \index{listener augmentation}\emph{listener
  augmentation} if
$k\reduction{\ops{A}_{n,\lsn,C}}k_0\reduction{\fn{tskel}}k_1$ and~$C$
listens for $\fn{inv}(t_1)$,
i.e.\ $C=\seq{\inbnd\fn{inv}(t_1),\outbnd\fn{inv}(t_1)}$.  If
$t_c=\enc{t_0}{t_1}$, then skeleton~$k_1$ is a \emph{listener
  augmentation} if
$k\reduction{\ops{A}_{n,\lsn,C}}k_0\reduction{\fn{tskel}}k_1$ and
$C=\seq{\inbnd t_1,\outbnd t_1}$.
\end{defn}

For regular augmentation and displacement, {\cpsa} removes solutions
that lead to skeletons that are less general than other solutions,
that is, when there is a homomorphism from a solution to the omitted
solution.

\begin{defn}[Cohort Member]\label{def:cohort member}
For unrealized node~$n$ in a skeleton~$k_0$, and a position~$p$
at~$n$, $k_0\reduction{n,p}k_1$ asserts that $k_1$ is a member of the
\index{cohort}cohort of~$k_0$, where~$k_1$ is derived using
contraction, regular augmentation, displacement, or listener
augmentation, and~$p$ is solved in~$k_1$ after~$k_0$ at~$n$.  For the
setwise cohort member reduction, $\{k_0\}\setreduction{n,p}\{k_1\mid
k_0\reduction{n,p}k_1\}$, when~$n$ is unrealized in~$k_0$, and~$p$ is
a critical position at~$n$.
\end{defn}

\begin{conj}[Critical Message Solved]
If $k_0\reduction{n_0,p_0}k_1 \reduction{n_1,p_1}\ldots
\reduction{n_{\ell-1},p_{\ell-1}}k_\ell$ is a sequence of cohort member
reductions, then for positive~$\ell$,~$p_0$ is solved in~$k_\ell$
after~$k_0$ at~$n_0$.
\end{conj}

\section{Augmentation}\label{sec:augmentation}

Let~$t_c$ be the critical message that demonstrates~$n$ is a test node
in skeleton~$k$.  For each triple~$(\sigma,r,C)$ that satisfies some
properties, there is a potential regular augmentation
with~$\comp{\reduction{\fn{tskel}}}%
{\comp{\reduction{\ops{A}_{n,r,C}}}{\reduction{\ops{S}_\sigma}}}$.
When successful, the message~$t$ in the last event of the added strand
is outbound, carries~$\sigma(t_c)$, but~$\sigma(t_c)$ is not carried
only within escape set~$\sigma(T_e)$ in~$t$, where
$T_e=\fn{esc}(k,n,t_c)$, the escape set.  Moreover, for every other
message~$t$ in the strand,~$\sigma(t_c)$ is carried only within escape
set~$\sigma(T_e)$ in~$t$.  The last event in the strand is called a
\index{transforming event}\emph{transforming event}, as this event no
longer protects the critical message, but events that precede it do.

{\cpsa} computes the parameters for a set of augmentation steps as
follows.  Suppose skeleton~$k=\skel_X(\_,P,\_,\_,\_,\_)$.  First,
compute the target messages, $T_t=\fn{targ}(t_c, T_e)$.  Next, for
each non-listener role $\role(C_r,N_r,U_r)\in P$ and each index~$h$ where
$C_r(h)=\outbnd t$, a transmission, do the following.

\begin{description}
\item[Create fresh variables:] Let~$\sigma_r$ be a sort preserving
  variable renaming, where the domain is the variables that occur in
  $\prefix{C_r}h$, and every variable in the range does not occur
  in~$X$ or in~$P$.

\item[Insert critical message:] For each message~$t'$ carried by~$t$,
  and each~$t_t\in \fn{targ}(t_c, T_e)$, consider most general
  unifiers~$\sigma'$ where, $\sigma'(t')=\sigma'(t_t)$ and
  $\sigma_r\unlhd\sigma'$.  (In other words,
  $\sigma'=\sigma_0\circ\sigma_r$ for some $\sigma_0$.)

\item[Ensure previous events do not transform:] For each $\sigma'$,
  find most general unifiers~$\sigma$ such that for $0\leq i<h$,
  $\sigma(t_c)$ is carried only within $\sigma(T_e)$ at $\sigma(C(i))$
  and $\sigma'\unlhd\sigma$.  The function \fn{cowt}, presented in
  Figure~\ref{fig:cowt}, performs the explorations, producing the
  substitutions $S'=\fn{cowt}(t_c,T_e,\prefix{C_r}{h},S)$.  Function
  $\fn{fold}$ is defined in Figure~\ref{fig:cows}.  Let $S_{r,h}$ be
  the set~$S'$ with non-most general unifiers removed.
\item[Ensure last event transforms:] For each $\sigma\in S_{r,h}$, if
  $\sigma(t_c)$ is not carried only with $\sigma(T_e)$ at
  $\sigma(C(h))$, try augmenting with parameters $n$, $r$,
  $\sigma\circ\prefix{C}{h}$, and~$\sigma$.
\end{description}

\begin{figure}
\begin{center}
\begin{tabular}{l}
$\fn{cowt}(t,T,C,S) ={}$\\
$\quad\bigcup_{\sigma\in S}\fn{cowt}_0(t,T,C,\sigma)$\\
\\
$\fn{cowt}_0(t,T,C,\sigma) ={}$\\
\quad \textbf{if} $\all{t}\pm t\in C\to\sigma(t)$ is {\cow} $\sigma(T)$ at
$\sigma(t')$ \textbf{then}\\
\qquad $\{\sigma\}$\\
\quad \textbf{else}\\
\qquad$\fn{cowt}(t,T,C,\fn{foldn}(t, T, C, \{\sigma\}))$\\
\\
$\fn{foldn}(t,T,\seq{},S)=S$\\
$\fn{foldn}(t,T,\seq{\pm t'}\append C,S)={}$\\
$\quad\fn{foldn}(t,T,C,\bigcup_{\sigma\in S}\fn{fold}(t,T,t',\sigma))$
\end{tabular}
\end{center}
\caption{The $\fn{cowt}$ Function}\label{fig:cowt}
\end{figure}

\begin{note}
For target terms to be the reasonable set for insertion of the
critical message, one must require that variables of sort message are
acquired.  This fact needs to be explained and noted as another reason
for the acquired variable constraint.
\end{note}

The following example shows the need for Item~\ref{item:targs} in the
definition of a solved critical position.  The protocol has two roles.

\[\begin{array}{r@{{}={}}l}
\init(a\colon\srt{N},n_1,n_2\colon\srt{D},k\colon\srt{A})&
\role(\seq{\outbnd\enc{n_1,\enc{a,n_2}{k}}{k},\inbnd\enc{a,n_1}{k}},
\emptyset,\emptyset)\\
\resp(n_1\colon\srt{D},m\colon\top,k\colon\srt{A})&
\role(\seq{\inbnd\enc{n_1,m}{k},\outbnd m},
\emptyset,\emptyset)
\end{array}\]
where we use~\srt{N} for sort \dom{name}, \srt{D} for sort \dom{data}
\srt{A} for sort \dom{akey}, and $\top$ for sort \dom{mesg} to save
space.  Notice the responder declares $m$ to be of sort \dom{mesg}.

The point-of-view skeleton $k_0$ is:
\[\begin{array}{r@{}ll}
\skel_Y(
&\seq{\init(a',n'_1, n'_2,k')}
&\mbox{Role map}\\
&\{\init(a',n'_1, n'_2,k'),\resp(n''_1,m'',k'')\},
&\mbox{Protocol}\\
&\seq{\seq{\outbnd\enc{n_1,\enc{a,n_2}{k}}{k},\inbnd\enc{a,n_1}{k}}},
&\mbox{Traces}\\
&\emptyset,
&\mbox{Node orderings}\\
&\{k^{-1}\},
&\mbox{Non-origination}\\
&\{n_1\})
&\mbox{Unique origination}
\end{array}\]
where $Y=a\colon\srt{N},n_1,n_2\colon\srt{D},k\colon\srt{A}$.

{\cpsa} finds the nonce test $(k_0,(0,1),\seq{0,1})$ with critical
message $n_1$.  The escape set $E=\fn{esc}(k_0, (0,1), n_1) =
\{\enc{n_1,\enc{a,n_2}{k}}{k}\}$.

{\cpsa} solves this test to produce skeleton $k_1$ with a displacement
that simply equates $n_1$ and $n_2$.  Thus,
$k_0\homomorphism{\phi,\sigma}k_1$ where $\phi=\seq{0}$ and
$\sigma=\{n_1\mapsto n_2\}$.

Consider Definition~\ref{def:critical position solved}, the definition
of critical position solved.  Item~\ref{item:anc} does not apply
because an ancestor of the image of the critical position does not
match~$E$.  Item~\ref{item:aug} does not apply because
$(k_1,(0,1),\seq{0,1})$ is a nonce test.  Item~\ref{item:dec} does not
apply because the key that protects the member of the escape set is
non-originating.  Finally, Item~\ref{item:enc} does not apply because
the critical message is not an encryption.

Item~\ref{item:targs} applies in this example because
\[\begin{array}{r@{{}={}}l}
\fn{targs}(n_1,E) & \{n_1,(n_1,\enc{a,n_2}{k})\}\\
\fn{targs}(\sigma(n_1),\sigma(E))&\{n_2,(n_2,\enc{a,n_2}{k}),(a,n_2)\}.
\end{array}\]
so
$\fn{targs}(\sigma(n_1),\sigma(E))\setminus\sigma(\fn{targs}(n_1,E))=\{(a,
n_2)\}$.

Once $n_1$ and $n_2$ have been equated, {\cpsa} easily finds the one
shape by adding a responder instance.  The shape corresponds to a
rather odd use of the initiator role.  The instance of that role picks
a fresh value for $n_1$ and then makes use of that value for its $n_2$
parameter.  See Appendix~\ref{chp:traces} for more on the relation
between roles and the programs they specify.

\chapter{Collapsing and Preconditioning}\label{chp:collapsing}

The input preskeleton is preconditioned before it is subjected to
authentication test solving.  The preskeleton is converted to a
skeleton and then collapsing is applied so as to ensure all shapes are
found.  Collapsing handles the case in which strands merged in the
input lead to shapes.

\begin{defn}[Collapsing]
Let $k_0$ and $k_1$ be two skeletons such that there are two
strands,~$s$ and~$s'$, and a most general unifier~$\sigma$ such that
$\sigma(\Theta_{k_0}(s)(j))\equiv \sigma(\Theta_{k_0}(s')(j))$ for all
$0\leq j<|\Theta_{k_0}(s)|$.  Then~$k_0$ \emph{collapses} to~$k_1$,
written $k_0\reduction{\fn{clp}}k_1$, if
$k_0\reduction{\ops{S}_\sigma}k\reduction{\ops{C}_{s,s'}}k'
\reduction{\fn{skel}}k_1$.
\end{defn}

\begin{defn}[Preconditioning]
For point-of-view preskeleton~$k_0$,~$k_0$ is preconditioned to~$k_1$,
written $k_0\reduction{\fn{pre}}k_1$, if
$k_0\reduction{\fn{skel}}k\mathbin{(\reduction{\fn{clp}})^\ast}k_1$.
\end{defn}

\chapter{Generalization}\label{chp:generalization}

The cohort reduction system produces a set of realized skeletons.
Generalization attempts to convert that set into a set of skeletons.
Not all possible cases are implemented due to performance issues, so
it is not uncommon to find a realized skeleton in the output of a run
of {\cpsa} that is not a shape.

\begin{conj}[Shape Completeness]
Without generalization, {\cpsa} produces a complete set of shapes
among the realized skeletons in its output.
\end{conj}

\begin{defn}[Generalize]
A skeleton~$k_0$ \emph{generalizes}
skeleton~$k_1$, written $k_1\reduction{<}_k k_0$, if
both~$k_0$ and~$k_1$ are realized,~$k_0$ and~$k_1$ are not isomorphic,
there is a homomorphism from a point-of-view skeleton~$k$ to~$k_0$, and
a strandwise injective homomorphism $k_0\mapsto k_1$.
\end{defn}

If skeletons are allowed to be isomorphic, we
write~$k_1\reduction{\leq}_k k_0$, and note that $\reduction{\leq}_k$
defines a partial ordering.  Therefore, there are maximal elements in
the partial ordering.  A shape associated with a preskeleton
is a maximally generalized realized skeleton derived from the
preskeleton.

\begin{defn}[Shape]
Let~$k_0$ be a preskeleton such that $k_0\reduction{\fn{pre}}k$ for
some skeleton~$k$, and let~$k_1$ be a realized skeleton such that
$k\mapsto k_1$.  Skeleton~$k_2$ is a \index{shape}\emph{shape}
of~$k_0$ if $k_1\reduction{\leq}_k k_2$, and~$k_2$ is maximal among
skeletons that generalize~$k_1$.
\end{defn}

There are four generalization reductions used to transform a realized
skeleton into its shapes: deletion, weakening, forgetting, and
separation.

\begin{defn}[Deletion]
Skeleton~$k_0$ \index{deletion}\emph{generalizes by deletion}
skeleton~$k_1$, written $k_1\reduction{\ops{D}_n}_k k_0$, if
$k_1\reduction{<}_k k_0$, $k_2\reduction{\fn{skel}}k_0$, and~$k_2$ is
the result of deleting node~$n$ in~$k_1$ and all of the nodes that
follow it in its strand.
\end{defn}

\begin{defn}[Weakening]
Skeleton~$k_0$ \index{weakening}\emph{generalizes by weakening}
skeleton~$k_1$, written $k_1\reduction{\ops{W}_{n,n'}}_k k_0$, if
$k_1\reduction{<}_k k_0$, $k_2\reduction{\fn{skel}}k_0$, and~$k_2$
is~$k_1$ except $\kprec{k_2}=(\kprec{k_1}\setminus\{(n,n')\})^\ast$.
\end{defn}

\begin{defn}[Forgetting]
Skeleton~$k_0$ \index{forgetting}\emph{generalizes by origination
  assumption forgetting} skeleton~$k_1$, written
$k_1\reduction{\ops{F}_t}_k k_0$, if $k_1\reduction{<}_k k_0$,
$k_2\reduction{\fn{skel}}k_0$, and~$k_2$ is~$k_1$ except
$U_{k_2}=U_{k_1}\setminus\{t\}$ and $N_{k_2}=N_{k_1}\setminus\{t\}$.
\end{defn}

Sometimes a more general skeleton can be found by replacing some
occurrences of one variable by a fresh variable.  For variable
separation, the location of an occurrence of a variable is defined
using a skeleton's instance.  Recall that in the external syntax,
strand~$s$ in skeleton~$k$ is described by an instance of the form
$\inst(r,h,\sigma)$.  (Instances are introduced in
Section~\ref{sec:instances}.)

\begin{defn}[Location]
Variable~$x$ is at \emph{location} $(s,y,p)$ in~$k$ if
the instance at $\Theta_k(s)$ is $\inst(r,h,\sigma)$ and
$x=\sigma(y)\termat p$.
\end{defn}

\begin{defn}[Variable Separation]
Skeleton~$k_0$ \index{variable separation}\emph{generalizes by
  variable separation} skeleton~$k_1$, written
$k_1\reduction{\ops{V}_t}_k k_0$, if $k_1\reduction{<}_k k_0$,
$k_2\reduction{\fn{skel}}k_0$, and~$k_2$ is~$k_1$ except~$t$ is a
variable that occurs in multiple locations in~$k_1$, and~$k_2$ is the
result of replacing~$t$ with a variable~$t_0$ of the same sort at a
proper subset of~$t$'s locations, where~$t_0$ occurs nowhere in~$k_1$.
\end{defn}

When separating a non-originating term, both the term and its clone
are non-originating.  When separating a uniquely originating term,
either the term or its clone is uniquely originating.

\begin{note}
What happens when separating~$t$ in~$k$ into~$t$ and~$t_0$, and
$\cn{ltk}(t,t)\in N_k$?  Should a skeleton~$k_0$ with
$\cn{ltk}(t,t_0)\in N_{k_0}$ be a candidate separation?
Currently, only skeletons~$k_1$ with $\cn{ltk}(t,t)\in N_{k_1}$
and $\cn{ltk}(t_0,t_0)\in N_{k_1}$ are considered.
\end{note}

\begin{defn}[Generalization]
The reduction~$\reduction{\fn{gen}}_k
=\bigcup_n\reduction{\ops{D}_n}_k\cup
\bigcup_{n,n'}\reduction{\ops{W}_{n,n'}}_k\cup
\bigcup_t\reduction{\ops{F}_t}_k\cup \bigcup_t\reduction{\ops{V}_t}_k$
is the \emph{generalization} relation.  For the setwise generalization
reduction, $\{k_0\}\setreduction{\fn{gen}}_k\{k_1\}$ when
$k_0\reduction{\fn{gen}}_k k_1$.
\end{defn}

The fact that each generalization reduction replaces a singleton with
just a singleton requires explanation.  It's simply a matter of
performance.  If all possibilities are considered, {\cpsa} run time
would become dominated by generalization.  Since generalization
failures do not interfere with producing a complete description of the
input, an approximation of the set of shapes is okay.

\begin{conj}[Generalization]
The relation~$\setreduction{\fn{gen}}_k$ is terminating.
\end{conj}

\section*{Discussion}
In~\cite{DoghmiGuttmanThayer07}, the shapes of a point-of-view
skeleton are said to be minimal, in the partial ordering induced by
injective homomorphism, among all realized homomorphic images of the
point-of-view skeleton.  Minimal corresponds to maximally generalized.
The need for origination assumption forgetting was not known
when~\cite{DoghmiGuttmanThayer07} was written.  Generalization by
variable separation uses non-carried positions, and in particular,
positions that traverse an atom edge.  Algebras in previous strand
space papers have no concept of a position that traverses an atom
edge, and therefore cannot be used to specify generalization by
variable separation.

Variable separation can be expensive when there are many possible ways
to separate variables.  The implementation simply truncates the search
when it grows too large.  The current implementation never attempts to
separate channel variables.

\chapter{Skeleton Reduction System}\label{chp:skeleton reduction}

Let reduction
$\twoheadrightarrow_k=(\setreduction{\fn{co}}\cup
\setreduction{\fn{gen}}_k)^+$.  This reduction system specifies the
{\cpsa} program.

\begin{conj}
The reduction~$\twoheadrightarrow_k$ is confluent.
\end{conj}

\begin{conj}[Soundness]\label{cnj:soundness}
Let~$k_0$ be a preskeleton and~$k$ be an unrealized skeleton such that
$k_0\reduction{\fn{pre}}k$.  Skeleton~$k_1$ is a shape of~$k_0$ if
$\{k\}\twoheadrightarrow_k K$, $k_1\in K$, and~$K$ is a normal form.
\end{conj}

The set of bundles denoted by preskeleton~$k$, $\sembrack{k}$ is
defined on Page~\pageref{def:preskeleton denotation}.

\begin{conj}[Completeness]\label{cnj:completeness}
Let~$k_0$ be a preskeleton and~$k$ be an unrealized skeleton such that
$k_0\reduction{\fn{pre}}k$.  For all~$K$ such that $\{k\}\twoheadrightarrow_k
K$, $\sembrack{k_0}=\bigcup_{k_1\in K}\sembrack{k_1}$.
\end{conj}

\chapter*{Acknowledgement}

Carolyn Talcott and Leonard Monk provided valuable feedback on drafts
of this document.

\appendix

\chapter{Penetrator Non-Origination Assumptions}\relax
\label{chp:penetrator non-origination}

Penetrator non-origination assumptions have been added as an extension
to the basic strand space theory.  An atom is \index{penetrator
  non-originating}\index{non-originating!penetrator}\emph{penetrator
  non-originating} in a bundle if it originates on no penetrator
strand, but each of its variables occurs in some strand's trace.

Penetrator non-origination assumptions can be used to model
passwords.  Several regular participants might know a password and
originate it in a run of a protocol, but an idealized password is one
the penetrator cannot guess.

A penetrator non-originating atom is similar to a non-originating
atom, except in that it can be carried.  There are two definitions
that require change.  Penetrator non-originating atoms must be added
to the avoidance set of Definition~\ref{def:avoid}.  When the message
at a critical position is an atom (see Definition~\ref{def:critical
  position}, Item~\ref{item:nonce, encryption, or channel}), instead of being
uniquely originating, and can also be penetrator non-originating.

\chapter{Programs Specified by a Role}\label{chp:traces}

Given the definitions in Chapter~\ref{chp:strand spaces and bundles},
a role can be viewed as an abstraction of a program, and a strand as
an abstraction of a run of a program.  But what program is specified
by a role?

Consider a role that contains the event $\inbnd\enc{t_0}{t_1}$.  If
the program has the decryption key~$\fn{inv}(t_1)$ before the message
is received, the program could decrypt the message and extract~$t_0$.
Alternatively, if the program has $\enc{t_0}{t_1}$, or has~$t_0$ and
the encryption key~$t_1$, it might check to see if the received
message is the same as the expected message, and abort the run if not.

Here is an example of when the second behavior is desired.  In
{\cpsa}, before hashing was part of the algebra, an encryption was
used to represent hashing.  The hash of~$t$, $\#t$, expanded to
$\enc{\mbox{``hash''},t}{h}$, where~$h$ was an asymmetric key known to
all, but no one knew~$h^{-1}$.  The tag ``hash'' was added to the
encryption so as to ensure a hash was never confused with other uses of
encryption.

A role does not specify a valid program if the only possible way of
interpreting the event~$\inbnd\#t$ is by using~$h^{-1}$ to decrypt
$\enc{\mbox{``hash''},t}{h}$.  The remainder of this section describes
how this class of specification errors is detected for the Basic Crypto
Algebra.

The behaviors associated with a trace depend on the set of messages
available initially.  The behaviors are specified by a data flow
relation, $\flow{T_0}{C}{T_1}$.  For trace~$C$, the relation
$\flow{T_0}{C}{T_1}$ asserts that when messages~$T_0$ are available
initially, there is a behavior of~$C$ that produces messages~$T_1$.

A derivation tree used to demonstrate $\flow{T_0}{C}{T_1}$ shows the steps
that enable the flow of data.  The tree can be linearized, and thus
specifies a sequential program that implements the role.

The data flow relation is defined with the aid of a data flow relation
for a sequence of events, $\flow{T_0}{C}{T_1}$.

$$\flow{T}{\seq{}}{T}\qquad
\infer{\flow{T_0}{\seq{\pm t}}{T}\quad\flow{T}{C}{T_1}}%
{\flow{T_0}{\seq{\pm t}\append C}{T_1}}$$

The $\flow{T_0}{\seq{\pm t}}{T_1}$ relation is defined using the
$\flow{T_0}{C}{T_1}$ relation.  An outbound message can be formed if
it is available initially
$$\infer{t\in T}{\flow{T}{\seq{+t}}{T}}$$
or if it can be formed by construction.
$$\infer{\flow{T}{\seq{+t_1,\ldots,+t_{n}}}{T}}%
{\flow{T}{\seq{+f(t_1,\ldots,t_{n})}}{T}}\quad
\left[\begin{array}{l}
f(t_1,\ldots,t_{n})\\
\mbox{not an atom}
\end{array}\right]$$

An inbound message makes atoms and acquired variables available.
$$\flow{T}{\seq{-t}}{T\cup\{t\}}\quad [\mbox{$t$ an atom or a variable}]$$
When the decryption key is available, the contents of the encryption
are also available.  Furthermore, the encryption can be sent in future
messages without access to its encryption key.
$$\infer{\flow{T_0}{\seq{+\fn{inv}(t_1)}}{T_0}\quad
  \flow{T_0}{\seq{-t_0}}{T_1}}{\relax
  \flow{T_0}{\seq{-\enc{t_0}{t_1}}}{T_1\cup\{\enc{t_0}{t_1}\}}}$$ A received
encryption that can be sent ensures the encryption agrees with
currently available terms and makes nothing new available.
$$\infer{\flow{T}{\seq{+\enc{t_0}{t_1}}}{T}}{\relax
  \flow{T}{\seq{-\enc{t_0}{t_1}}}{T}}$$

Consider an operation $f$ other than the encryption operation.  The
order in which messages that occur in a message constructed using~$f$
are made available may determine if the decryption key of an
encryption is available.  All possible orderings must be explored.
Let~$\pi_n$ be a permutation on the domain of a sequence of
length~$n$.
$$\infer{\flow{T_0}{\seq{-t_1,\ldots,-t_{n}}\circ\pi_n}{T_1}}%
{\flow{T_0}{\seq{-f(t_1,\ldots,t_{n})}}{T_1}}
\left[\begin{array}{l}
f(t_1,\ldots,t_{n})\\
\mbox{not an atom}
\end{array}\right]$$

The data flow relation is used to find initial sets of atoms that are
compatible with some behavior of a trace that produces messages.

\begin{defn}[Trace Parameters]
The set of atoms~$T_0$ are \index{parameters}\emph{parameters} of
trace~$C$ if $\flow{T_0}{C}{T_1}$ for some~$T_1$, and~$T_0$ is
minimal, that is for all~$T'_0$ such that $\flow{T'_0}{C}{T_1}$,
$T'_0\not\subset T_0$.
\end{defn}

The role $\seq{\outbnd\enc{a,n}{K_b},\inbnd\enc{n}{K_a}}$ has two sets
of parameters, $\{a,n,K_b,K^{-1}_a\}$ and $\{a,n,K_b,K_a\}$.  See
Figure~\ref{fig:ambigrole} for two examples of programs that implement
the role, using distinct parameter sets.

\begin{figure}
\begin{center}
\begin{tabular}{l@{\hspace{8mm}}l}
$\cn{proc}(a,n,K_b,K^{-1}_a)$ & $\cn{proc}(a,n,K_b,K_a)$\\
$\quad \cn{send}(\enc{a,n}{K_b});$ & $\quad \cn{send}(\enc{a,n}{K_b});$\\
$\quad x_0\gets \cn{recv}();$ & $\quad x_0\gets \cn{recv}();$\\
$\quad x_1\gets \cn{decrypt}(x_0,K^{-1}_a);$ & $\quad x_1\gets \enc{n}{K_a};$\\
$\quad x_1 \neq n \to \cn{fail};$ & $\quad x_0 \neq x_1 \to \cn{fail};$ \hspace{8mm}\\
$\cn{end}$ & $\cn{end}$
\end{tabular}
\end{center}
\label{fig:ambigrole}
\caption{Two programs that implement the role $\seq{\outbnd\enc{a,n}{K_b},\inbnd\enc{n}{K_a}}$.}
\end{figure}

The {\cpsa} distribution contains a program that computes the set of
parameters of a role.  It was used to find an error in a role's use of
hashing as described at the beginning of this section.  The role in
question is the verifier role that is part of an attestation
protocol~\cite{CokerEtAl11}.  In an earlier version of the role,
every set of parameters included the decryption key of the encryption
used as a hash.  The role was so complicated that inspection did not
reveal the error.

Note that this section is specific to the basic crypto algebra in the
sense that all non-atomic operations are assumed to be constructable,
and encryption has the specific de-construction recipe shown.  Although
carried positions were not mentioned in this section, the inference system
specifies the same concept.  In fact, in future work we hope to show that
the inference system is all one needs to define, and that which positions
are carried, which are protected, and which sorts are atoms can be defined
in terms of the inferences available.

\chapter{Shape Analysis Sentences}\label{chp:logic}

For each point-of-view skeleton and its shapes found by {\cpsa}, there
is a formula in a language of order-sorted first-order logic called
a \index{shape analysis sentence}\emph{shape analysis sentence}.  The
sentence has a special form,
$\all{X}(\Psi\supset\bigvee_i\some{Y_i}(\Delta_i\wedge\Phi_i))$,
where~$\Psi$, $\Delta_i$, and~$\Phi_i$ are conjunctions of atomic
formulas and~$X$ and~$Y_i$ are variable sets.  This fragment of
first-order logic is called coherent logic.  Formula~$\Psi$ describes
the point-of-view skeleton~$k_0$.  For each homomorphism to a shape,
$k_0\homomorphism{\delta_i}k_i$, formula~$\Delta_i$ describes the
structure preserving maps~$\delta_i$, and the shape~$k_i$ is described
by~$\Phi_i$.

An interpretation of a shape analysis sentence is a skeleton.  If
{\cpsa} finds all of the shapes and the homomorphisms associated with
a point-of-view skeleton, the analysis' sentence is satisfied in all
realized skeletons.  Let~$\Sigma$ be a shape analysis sentence
and~$\Psi$ be a security goal.  If $\Sigma\supset\Psi$ is a theorem in
order-sorted first-order logic, then~$\Psi$ is satisfied in all
realized skeletons and its protocol achieves this goal.

Shape analysis sentences are closely related to security goals
in~\cite{guttman09}, and were motivated by that work.  This material
was extracted from~\cite{ramsdell12}, the paper that introduced shape
analysis sentences.

\section{Security Goals}\label{sec:security goals}

The signature for security goal terms extends the one used for the
underlying message algebra with sort \srt{Z}.  Variables of this sort
denote strands.

Security goals make use of protocol specific and protocol independent
predicates.  For each role $\role_Y(C,N,U)\in P$, there are strand length
predicates and strand parameter predicates.  There are $|C|$ unary
length predicates $P[r,h]:\srt{Z}$, with $1\leq h\leq|r|$.  Relative
to skeleton~$k$, $P[r,h](z)$ asserts that strand~$z$ in~$k$ is an
instance of~$r$ and has a length of at least~$h$.  For each
variable~$x:S$ that occurs in~$C$, there is a binary parameter
predicate $P[r,x]:\srt{Z}\times{}S$.  Relative to skeleton~$k$,
$P[r,x](z,t)$ asserts that strand~$z$ in~$k$ is an instance of~$r$ in
which~$x$ is instantiated as~$t$.

For each base sort~$B$, there are unary predicates $\cn{non}\colon B$
and $\cn{uniq}\colon B$.  $\cn{non}(t)$ asserts~$t$ is non-originating
in~$k$ and $\cn{uniq}(t)$ asserts~$t$ uniquely originates in~$k$.

Let $m$ be the length of the longest role in~$P$.  There are $m^2$
precedence predicates $\cn{prec}[i,j]:\srt{Z}\times\srt{Z}$ for $0\leq
i,j< m$.  $\cn{prec}[i,j](x,y)$ asserts that node $(x,i)$ is before
node $(y,j)$ in~$k$.  There are $3m$ origin predicates
$\cn{uniq-at}[i]:B\times\srt{Z}$, with $0\leq i< m$ and $B$ as before.
$\cn{uniq-at}[i](t,z)$ asserts that~$t$ uniquely originates in~$k$ at
node $(z,i)$.  The predicate \cn{false} has arity zero and, of course,
equality is binary.

To improve the readability of formulas to follow, we write
$\cn{prec}(x,i,y,j)$ for $\cn{prec}[i,j](x,y)$ and $\cn{uniq-at}(t,z,i)$
for $\cn{uniq-at}[i](t,z)$.

Let $k=\skel_X(P,I,\prec,N,U)$ and let~$\prec^-$ be the transitive
reduction of~$\prec$.  Recall that~$\Theta_X$ is the strand space
defined by~$I$, see Section~\ref{sec:instances}.  We
define~$\form(k)=(Y,\Phi)$, where~$\Phi$ is $k$'s skeleton formula,
and~$Y$ is the formula's variable set.  For~$\form(k)=(Y,\Phi)$, the
variable set~$Y$ is~$X$ augmented with a fresh variable of
sort~\srt{Z} for each strand in~$k$.  Let $Z(s)$ be the strand
variable associated with strand $s\in\sdom(I)$.

The formula~$\Phi$ is a conjunction of atomic formulas composed as
follows.
\begin{itemize}
\item For each $s\in\sdom(I)$, let $I(s)=\cn{i}(r,h,\sigma)$ and
  $z=Z(s)$.  Assert $P[r,h](z)$.  Let $r=\role(C,N,U)$.  For each
  variable $x\in\svars(\prefix{C}{h})$ and term $t=\sigma(x)$, assert
  $P[r,x](z,t)$.
\item For each $(s,i)\prec^-(s',i')$ such that $s\neq s'$, assert
  $\cn{prec}(Z(s),i,Z(s'),i')$.
\item For each $t\in N$, assert $\cn{non}(t)$.
\item For each $t\in U$ and $(s,i)\in\orig_k(t)$, assert
  $\cn{\uniqat}(t,Z(s),i)$.
\item For each $t\in U$ without a point of origination in~$k$, assert
  $\cn{uniq}(t)$.
\end{itemize}

Given a set of homomorphisms $\delta_i\colon k_0\mapsto k_i$, its shape
analysis sentence is
\begin{equation}\label{eqn:node-oriented shape sentence}
\sent(\delta_i\colon k_0\mapsto k_i)=\all{X_0}(\Phi_0\supset
\bigvee_i\some{X_i}(\Delta_i\wedge\Phi_i)),
\end{equation}
where $\form(k_0)=(X_0,\Phi_0)$.  The same procedure produces~$X_i$
and~$\Phi_i$ for shape~$k_i$ with one proviso---the variables in
$X_i$ that also occur in~$X_0$ must be renamed to avoid trouble while
encoding the structure preserving maps~$\delta_i$.

The structure preserving maps~$\delta_i=(\phi_i,\sigma_i)$ are encoded
in~$\Delta_i$ by a conjunction of equalities.  Map~$\sigma_i$ is coded
as equalities between a message algebra variable in the domain
of~$\sigma_i$ and the term it maps to.  Map~$\phi_i$ is coded as
equalities between strand variables in~$\Phi_0$ and strand variables
in~$\Phi_i$.  Let~$Z_0$ be the strand variables freshly generated
for~$k_0$, and~$Z_i$ be the ones generated for~$k_i$.  The
strand mapping part of~$\Delta_i$ is
\[\bigwedge_{s\in\sdom(Z_0)}Z_0(s)=Z_i(\phi_i(s)).\]

\begin{figure}
$$\begin{array}{l}
\all{a_0,b_0\colon\srt{A}, s_0\colon\srt{S}, d_0\colon\srt{D},
z_0\colon\srt{Z}}(\\
\quad\resp_2(z_0)\wedge\resp_{a}(z_0,a_0)\wedge
\resp_{b}(z_0,b_0)\\
\qquad{}\wedge
\resp_{s}(z_0,s_0)\wedge
\resp_{d}(z_0,d_0)\\
\qquad{}\wedge\cn{non}(a_0^{-1})\wedge
\cn{non}(b_0^{-1})\\
\quad\supset\\
\quad\some{a_1,b_1,b_2\colon\srt{A}, s_1\colon\srt{S},
  d_1\colon\srt{D}, z_1,z_2\colon
 \srt{Z}}(\\
\qquad z_0=z_1\wedge a_0=a_1\wedge b_0=b_1\wedge s_0=s_1\wedge
d_0=d_1\\
\qquad\quad\wedge\resp_2(z_1)\wedge
\resp_{a}(z_1,a_1)\wedge
\resp_{b}(z_1,b_1)\\
\qquad\quad\wedge
\resp_{s}(z_1,s_1)\wedge
\resp_{d}(z_1,d_1)\\
\qquad\quad\wedge\init_1(z_2)\wedge\init_a(z_2,a_1)\wedge
\init_b(z_2,b_2)\wedge
\init_s(z_2,s_1)\\
\qquad\quad\wedge\cn{\uniqat}(s_1,z_2,0)\wedge
\cn{prec}(z_2,0,z_1,0)\\
\qquad\quad\wedge\cn{non}(a_1^{-1})\wedge
\cn{non}(b_1^{-1})))
\end{array}$$
\caption{Shape Analysis Sentence for Blanchet's
  Protocol}\label{fig:blanchet's shape analysis sentence}
\end{figure}

The shape analysis sentence for the second analysis of Blanchet's
Simple Example Protocol in Section~\ref{sec:blanchet's simple example
  protocol} is displayed in Figure~\ref{fig:blanchet's shape analysis
  sentence}.  The strand length predicate $P[r,h](z)$ is written
$r_h(z)$, and the strand parameter predicate $P[r,x](z,t)$ is written
$r_x(z,t)$ with the protocol left implicit.

\paragraph{Semantics of Skeleton Formulas.}

Let $k=\skel_X(P,I,\prec,N,U)$ and $\Theta_X$ be its strand space.
When formula~$\Phi$ is satisfied in skeleton~$k$ with order-sorted
variable assignment $\alpha$, we write $k,\alpha\models\Phi$.  For
$x:S\in X$, $\alpha(x)$ is in the carrier set of $\algbara_X$ for
sort~$S$.  For $x:\srt{Z}$, $\alpha(x)\in\sdom(\Theta_X)$.  We
write~$\bar\alpha$ when~$\alpha$ is extended to terms in the obvious
way.  When sentence~$\Sigma$ is satisfied in skeleton~$k$, we write
$k\models\Sigma$.

\begin{itemize}
\item $k,\alpha\models P[r,h](z)$ iff $h\leq|\Theta_X(s)|$, and for
  some~$\sigma$,
  \[\prefix{\Theta_X(s)}{h}=\sigma\circ\prefix{C}{h}\]
  where $s=\alpha(z)$ and $r=\role(C,N,U)$.

\item $k,\alpha\models P[r,x](y,t)$ iff
  for some~$\sigma$ with $\sigma(x)=\bar\alpha(t)$,
  \[\prefix{\Theta_X(s)}{h}=\sigma\circ\prefix{C}{h}\]
  where $s=\alpha(z)$, $r=\role(C,N,U)$, and $h$ is the smallest
  $\ell$ such that $x$ occurs in $\prefix{C}{\ell}$.

\item $k,\alpha\models\cn{prec}(x,i,y,j)$ iff
  $(\alpha(x),i)\prec(\alpha(y),j)$.
\item $k,\alpha\models\cn{non}(t)$ iff $\bar\alpha(t)\in N$.
\item $k,\alpha\models\cn{uniq}(t)$ iff $\bar\alpha(t)\in U$.
\item $k,\alpha\models\cn{\uniqat}(t,z,i)$ iff $\bar\alpha(t)\in U$ and
  $(\alpha(z),i)\in\orig_k(\bar\alpha(t))$.
\item $k,\alpha\models y=z$ iff $\bar\alpha(y)=\bar\alpha(z)$.
\end{itemize}

Let~$\Sigma$ be a shape analysis sentence and~$\Psi$ be a security
goal.  If $\Sigma\supset\Psi$ is a theorem in order-sorted first-order
logic, then~$\Psi$ is satisfied in all realized skeletons and its
protocol achieves this goal.

\paragraph{Channel Formulas.}

This appendix has not been updated to reflect the inclusion of
channels into {\cpsa}, but the changes made are straightforward.
There are two new unary predicate symbols, \cn{auth} and \cn{conf}, that
are similar to \cn{non} except their argument must be a variable of sort
\dom{chan}.  For skeleton $k$ and variable assignment $\alpha$,
$\cn{auth}(c)$ is satisfied iff $\alpha(c)\in\auth(k)$, and
$\cn{conf}(c)$ is satisfied iff $\alpha(c)\in\conf(k)$.

\chapter{Facts and Rules}\label{chp:facts and rules}

Support for facts and rules was introduced in version 4.1 of {\cpsa}.

Each skeleton includes a small database of facts.  A fact is a named
relation among fact terms.  A fact term\index{fact term} is either a
strand of the skeleton or an algebra term.  For example, in
skeleton~$k$, a user may want to note that strand 0 owns the private
key for~$a$ by asserting fact $(\cn{owns},\seq{0, a^{-1}})$.

Each protocol includes a small collection of rules.  A rule is a
sentence in the goal language presented in Section~\ref{sec:security
  goals}.  A rule is an axiom added to a protocol.  {\cpsa} uses the
axiom as a rewrite rule to derive zero or more new skeletons from a
skeleton produced during a cohort step.

\bibliography{cpsa}
\bibliographystyle{plain}

\printindex

\end{document}
